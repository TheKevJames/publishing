\documentclass[12pt]{article}
\usepackage{amsmath,amssymb,parskip,custom}
\usepackage[margin=1in]{geometry}

\begin{document}

\title{ECE 105 - Physics of Electrical Engineering 1}
\author{Kevin Carruthers}
\date{\vspace{-2ex}Fall 2012}
\maketitle\HRule

\section*{Forces and Motion}
{\bf Force} is a vector, and therefore includes direction. For any vector $a$, $-a$ has the same magnitude but opposite direction.

\subsection*{Coordinate Systems}
Given $\vec{AB}$ we can find $A$ or $B$'s position based on the position of the other one \[ \vec{O_B} = \vec{O_A} + \vec{AB} \] for any $\vec{O_x}$ is the location of $x$ relative to the origin.

\subsection*{Components}
We can break any vector into {\bf components} by finding the angle between it and the plane we want to model it off of.

Example: for $\vec{A} = 6 @ 20^\circ$, we can find it's components with relation to the standard $x$-$y$ plane with
\begin{align*}
\vec{A_y} &= \vec{A}\cos 20^\circ\\
\vec{A_x} &= \vec{A}\sin 20^\circ
\end{align*}

\subsection*{Constant Acceleration}
\begin{align*}
\vec{a} &= \frac{\Delta\vec{v}}{\Delta t}\\
\vec{a} &= \frac{\vec{v_f}-\vec{v_i}}{\Delta t}\\
\vec{a}\Delta t &= \vec{v_f}-\vec{v_i}\\
\vec{v_f} &= \vec{v_i} + \vec{a}\Delta t
\end{align*}

For the position vector $\vec{d}, \vec{d_f} = \vec{d_i} + \vec{v_i}\Delta t + \half \vec{a}(\Delta t)^2$

$\vec{v_f}^2 = \vec{v_i}^2 + 2\vec{a}\bigl(\Delta\vec{d}\bigl)$

\subsection*{Relative Motion}
For any three objects $a, b,$ and $c$ \[ \vec{v_{ca}} = \vec{v_{cb}} + \vec{v_{ba}} \] read "the velocity of $c$ with respect to $a$ is equal to the velocity of $c$ with respect to $b$ \emph{plus} the velocity of $b$ with respect to $a$.

\subsection*{Circular Motion}
$a_c = \frac{v^2}{r}$\\
$\theta \approx \frac{\bigtriangleup x}{r}$ for very close points.\\
For circle with center $O$ and radius $r_0, r_1...$ connected to object on circumferance with velocity $\vec{v_0}, \vec{v_1}...$ tangent to circumferance, $\vec{a} = \frac{\vec{v_1} - \vec{v_0}}{t}$. For arc length between object (at different times $t_0, t_1...$) $s$, $\theta = \frac{s}{r}$. For small $\theta \ll 1, |\vec{v_1} - \vec{v_0}| = \theta\vec{v}.$\\
$|\vec{a}| = \frac{\vec{v}\theta}{t}$\\
$\vec{a}$ is perpendicular to $\vec{v}$\\
$\vec{v} = \frac{r_1-r_0}{t} = \frac{r\theta}{t}$\\
$\therefore \vec{a} = \frac{\vec{v}\theta}{r\frac{\theta}{\vec{v}}} = \frac{\vec{v^2}}{r}$

\begin{align*}
s &= r\theta\\
\frac{ds}{dt} = v &= r\frac{d\theta}{dt}\\
&= r\omega\\
\omega &= \frac{d\theta}{dt}\\\\
a &= \frac{v^2}{r}\\
&= \frac{(r\omega)^2}{r}\\
&= r\omega ^2
\end{align*}

\subsection*{Types of Forces}
A force is a push or pull interaction between two objects, reponsible for changing motion.

\subsubsection*{Springs}
When unstretched, no {\bf spring forces} exist. When a string is pushed from equilibrium, its spring force pushes back toward equilibrium.
\[ F_s = -k\Delta x \]

\subsubsection*{Tension}
The {\bf tension force} pulls an object toward a rope and a rope toward an object. Ropes can never push.

\subsubsection*{Normal}
The {\bf normal force} "pushes back" against other objects via molecular electromagnetism. It is always perpendicular to the surface for any surface-to-surface contact. Technically, it is a type of spring force.

\subsubsection*{Friction}
{\bf Friction} is the interaction between an object and a surface. It is a real force which acts opposite the direction of sliding, and is always tangent to surface.

$f \propto N$ is an experimental fact. $f = \mu N$, where $\mu$ is the coefficient of friction. $\mu$ is dependant on the type of objects and must be determined experimentally.

{\bf Kinetic friction} is when objects are sliding relative to each other and {\bf static friction} is when objects are not yet sliding \[ f_s \leq \mu_s N \]

Example: A 50kg person is in a 1000kg elevator at rest. When the elevator begins to rise, the person notices her weight is 600N. How far does the elevator move in 3s?

\begin{align*}
\Sigma\vec{F} &= m\vec{a}\\
\vec{a} &= \frac{\vec{F_n}-mg}{m}\\
&= \frac{600-50g}{50}\\
&= 2.2{\rm m/s}^2\\\\
d &= \vec{v_i}t + \half \vec{a}t^2\\
&= 0 + \half (2.2)9\\
&= 9.9{\rm m}
\end{align*}

\section*{Energy}
An object can be said to have a total {\bf energy} equal to the sum of the various forms of energy it may posess.

\subsection*{Kinetic Energy}
The {\bf kinetic energy} of an object is determined by its mass and velocity \[ K = \frac{mv^2}{2} \]

For any object with a changing velocity
\begin{align*}
v_f^2 &= v_i^2 + 2ad\\
\half mv_f^2 &= \half mv_i^2 + mad\\
K_f &= K_i + \Sigma\vec{F}d\\
\Delta K &= \Sigma\vec{F}d
\end{align*}

\subsection*{Potential Gravitational Energy}
{\bf Potential gravitational energy} is a measure of stored energy of an object based on its height. It is essentially non-sensical to determine an object's "absolute" potential gravitational energy, thus we often simply solve for the difference in energy.

For a distance $h_f$ above a reference height $h_i$ \[ U_g = mg(h_f-h_i) \] thus if an object moves from $h_i$ to $h_f$
\begin{align*}
\Delta U_g &= U_{gf} - U_{gi}\\
&= mgh_f - mgh_i\\
&= mg\Delta h
\end{align*}

\subsection*{Spring Energy}
A {\bf spring's energy} is based on its spring constant $k$ and how far it is compressed from its equilibrium point \[ U_s = \frac{kx^2}{2} \]

\subsection*{Collisions}
If a {\bf collision} is isolated, then energy is conserved. Elastic collisions also conserve energy. For all real or inelastic collisions, energy is lost.

\subsection*{Work}
Just as energy is a way of keeping track of motion, {\bf work} is a mechanical means for transfering energy equal to the applied force multiplied by the distance it operates along \[ dW = \vec{F}\vec{ds} \]

It can be used to compute the change in energy of a system between two states, as the total work done by non-conservative forces (ie friction) will be equal to the work done by conservative forces (ie gravity, springs, motion)

For a system involving friction, motion, gravity, and a spring, we have \[ \Delta E_{th} = \Delta K + \Delta U_g + \Delta U_s \] or, if we compute the value of the thermal work done by friction as energy (using $U_f = \mu Nd$, where $d$ is the distance during which the object undergoes friction), we get \[ 0 = \Delta K + \Delta U_g + \Delta U_s + \Delta U_f \]

\section*{Rotation (of a non-deformable, rigid bodied object)}
For any point on an object in {\bf circular rotation}
\begin{align*}
\omega &= \frac{d\theta}{dt}\\
s &= r\theta\\
v &= \frac{ds}{dt} = \frac{rd\theta}{dt} = r\omega\\
a_r &= \frac{v^2}{r} = r\omega^2\\
a_t &= \frac{dv}{dt} = \frac{rd\omega}{dt} = r\alpha
\end{align*}

where $\omega$ is the angular frequency, $s$ is the arc length of a circle, and $\alpha$ is the angular acceleration

\subsection*{Centre of Mass}
For a uniform mass distribution, the {\bf centre of mass} is in the geometric centre. Otherwise \[ x_{centre} = \frac{m_1x_1 + ... + m_nx_n}{m_1 + ... + m_n} \] Gravity acts as if all the mass is located at the centre of mass.

\subsection*{Rotational Energy}
\[ E_k = \frac{mv^2}{2} = \frac{mr^2\omega^2}{2} = \frac{I\omega^2}{2} \] where $I$ is the moment of inertia.

\subsection*{Moment of Inertia}
\[ I = m_1x_1^2 + ... + m_nx_n^2 \]

For a thin rod of length $L$ and uniform mass $m$, \[ I = \frac{mL^2}{12} \]
For a filled ring of radius $r$ and uniform mass $m$ (regardless of length, ie. cylinders) \[ I = \frac{mr^2}{2} \]
For a hollow ring of radius $r$ and uniform mass $m$ (regardless of length, ie. hollow cylinders) \[ I = mr^2 \]
For a filled sphere of radius $r$ and uniform mass $m$ \[ I = \frac{2mr^2}{5} \]
For a hollow sphere of radius $r$ and uniform mass $m$ \[ I = \frac{2mr^2}{3} \]

To find a "new" moment of inertia, where $h$ is the distance to the new pivot \[ I = I_0 + mh^2 \]

\subsection*{Torque}
{\bf Torque} is a measure of how much a given applied force "wants" to rotate an object, where $r$ is the direction from an object to its pivot and $\theta$ is the angle between $r$ and the applied force $F$ \[ \tau = rFsin\theta \]

\section*{Static Equilibrium}
Any object in {\bf static equilibrium} undergoes no motion at all. \[ \Sigma F = \Sigma\tau = 0 \]

\section*{Rotational Dynamics}
\[ \Sigma\tau = I\alpha \]

\section*{Oscillations}
As {\bf oscillation} is a periodic motion about an equilibrium position.

\subsection*{Simple Harmonic Motion}
Any object in {\bf simple harmonic motion} follows a sinusoidal shape in terms of its distance from its equilibrium position. Note: \[ \omega = 2\pi f \]

\begin{align*}
x &= A\cos(\omega t + \phi)\\
v &= -A\omega\sin(\omega t + \phi)\\
\end{align*}

As you can see \[ v_{{\rm max}} = A\omega \]

\subsection*{Oscillation Dynamics}
\begin{align*}
x(t) &= A\cos(\omega t+\phi)\\
v(t) &= -Aw\sin(\omega t+\phi)\\
a(t) &= -A\omega^2\cos(\omega t+\phi)\\
a &= -\omega^2x\\
a_{max} &= -A\omega^2
\end{align*}

\subsection*{Simple Harmonic Motion Dynamics}
\begin{align*}
x &= A\cos(\omega t + \phi)\\
\omega &= 2\pi f = \frac{2\pi}{T}\\
v &= -A\omega\sin(\omega t + \phi)\\
a &= -A\omega^2\cos(\omega t + \phi)\\
\frac{{\rm d}^2x}{{\rm dt}^2} &= -\omega^2x
\end{align*}

Therefore, if \[ a = -Cx \] we know that a solution of x is \[ x = A\cos(\sqrt{C}t + \phi) \]

For an ideal spring where \[ Fs = -kx \] by dividng by mass we get \[ \omega = \sqrt{\frac{k}{m}} \]

\subsection*{Simple Pendulum}
For a {\bf simple pendulum} \[ a_r = \frac{v^2}{l} \] and \[ a_t = \alpha l \]

\begin{align*}
mg\sin\theta &= m\alpha l\\
\alpha &= \frac{g}{l}\sin\theta\\
\frac{{\rm d}^2\theta}{{\rm d}t^2} &= \alpha
\end{align*}

If $\theta << 1$ then $\sin\theta = \theta$

\subsubsection*{Physical Pendula}
For a {\bf physical pendulum}, the centre of mass is the location where gravity acts, and thus we have
\begin{align*}
\Sigma\tau &= I\alpha\\
mgx\sin\theta &= I\alpha\\
\theta^{\prime\prime} &= \frac{mgx}{I}\sin\theta\\
&\approx \frac{mgx}{I}\theta\\
\omega &= \sqrt{\frac{mgx}{I}}
\end{align*}

This tends to \[ \omega = \sqrt{n\frac{g}{l}} \] where $n$ is some real number.

\subsubsection*{Energy Conservation}
For any object in simple harmonic motion, {\bf energy must be conserved}. Thus we have
\begin{align*}
E &= \half I\omega_{max}^2\\
&= mgh
\end{align*}

\section*{Waves}
{\bf Waves} are physical areas of increased or decreased energy, which travel in simple harmonic motion. They can be visuallised on a horizontal line with regular "humps".

\subsection*{Wave Propogation}
The {\bf propogation} of a wave is the direction in which it travels. The form of the wave does not change as it propogates. A wave can be modelled by the equation \[ y = f(x\pm vt) \] travelling tot he left/right (plus/minus) where $y = f(x)$ is the equation of the wave.

We can find the transverse travelling wave by assuming the shape is preserved.

\subsection*{Harmonic Waves}
{\bf Harmonic waves} have a sinsusoidal form.

For a stationary harmonic wave, we have
\begin{align*}
y = f(x) &= A\sin(2\pi\frac{x\pm vt}{\lambda}+\phi)\\
&= A\sin(\frac{2\pi x}{\lambda} - \omega t + \phi)\\
&= A\sin(kx - \omega t + \phi)
\end{align*}
where \[ k = \frac{2\pi}{\lambda} \]

Thus the speed of any particle on the wave (in the up/down direction, where $x$ is constant) is \[ v = -A\omega\sin(kx - \omega t + \phi) \]

\subsection*{Waves on a String}
The velocity of a {\bf wave on a string} is based solely on the properties of the string. We define the mass density as $\mu = \frac{m}{l}$ so that \[ v = \sqrt{\frac{T}{\mu}} \] where $T$ is the tension in the string.

\subsection*{Laws of Superposition}
For $y_1 = f_1(x_1t)$ and $y_2 = f_2(x_2t)$ \[ y(x,t) = f_1(x_1,t) + f_2(x_2,t) \]

The {\bf superposition} of two waves is {\bf constructive} if it results in a larger amplitude and {\bf destructive} if it results in a small amplitude.

Consider two harmonic waves which are identical except for a phase shift travelling in the same direction on the same string.
\end{document}
