\documentclass[12pt]{article}
\usepackage{amsmath,amssymb,parskip,custom,graphicx}
\usepackage[margin=1in]{geometry}

\begin{document}

\title{CS 137 - Programming Principles}
\author{Kevin Carruthers}
\date{\vspace{-2ex}Fall 2012}
\maketitle\HRule

\section*{Outline}
We will be covering
\begin{enumerate}
\item Basic C programming concepts
\begin{itemize}
\item Variables
\item Integers, chars, expression evaluation
\item Conditionals
\item Loops (do, do while, for)
\end{itemize}
\item Functions, parameters, recursion
\item Arrays and pointers
\item Structures
\item Sorting, searching, time and space complexity
\end{enumerate}

\section*{Absolute Basics}
\begin{verbatim}
// import standard input/output library
#include <stdio.h>

// main function returns and integer and takes no (void) parameters
int main(void) {
  // print formatted "Hello, World!"
  printf("Hello, World!");
  
  // returns 0, ie successful completion
  return 0;
}
\end{verbatim}

\section*{Inputs}
\begin{verbatim}
#include <stdio.h>

int main(void) {
  // declare variables
  int a,b,r;
  
  // take two integers as input, assign them to a and b. %d is for integers
  scanf("%d,%d",&a,&b);
  
  // so long as b is non-zero...
  while(b) {
    // r is the remainder of a divided by b
    r = a % b;
    a = b;
    b = r;
  }
  printf("%d\n",a);

  return 0;
}
\end{verbatim}

\section*{Integer Data Types}
\begin{table}[ht]
\centering
  \caption{Types of integers and their sizes and ranges.}
  \begin{tabular}{r|ll}
  Type & Size & Range \\ \hline 
  unsigned char & 1 byte & $[0,2^8-1]$ \\ 
  char   & 1 byte & $[-2^{8-1},2^{8-1}]$ \\ 
  unsigned short & 2 bytes & $[0,2^{16}-1]$ \\ 
  short & 2 bytes & $[-2^{16-1},2^{16}]$ \\ 
  unsigned int & 4 bytes & $[0,2^{32}-1]$ \\ 
  int & 4 bytes & $[-2^{32-1},2^{32-1}]$ \\ 
  long & 4/8 bytes & $[-2^{64-1},2^{64-1}$ \\ \hline
  \end{tabular}
\end{table}

\section*{Logic}
False is denoted by zero, true is denoted by non-zero (traditionally one). The logical operators are
\begin{itemize}
\item NOT (\code{!})
\item OR (\code{||})
\item AND (\code{\&\&})
\end{itemize}

\subsection*{DeMorgan's Identities}
\begin{itemize}
\item \begin{verbatim}!(P && Q) == !P || !Q\end{verbatim}
\item \begin{verbatim}!(P || Q) == !P && !Q\end{verbatim}
\end{itemize}

\section*{Loops}
\begin{itemize}
\item \begin{verbatim}while(expression)
  statement // executes at least 0 times\end{verbatim}
\item \begin{verbatim}do
  statement
while(expression); // executes at least 1 time\end{verbatim}
\end{itemize}

\section*{Functions}
\begin{verbatim}
// create function gcd which can be called with gcd(x,y) and returns an integer answer
int gcd(int a, int b) {
  int r = a % b;
  while(r) {
    a = b;
    b = r;
    r = a % b;
  }
  
  return b;
}

int main() {
  // print the result of a function call with arguments 806 and 338
  printf("%d\n",gcd(806,338));
  
  return 0;
}
\end{verbatim}

\begin{verbatim}
// include boolean library (defines boolean type with true and false)
#include<stdbool.h>
#include<stdio.h>

bool isLeap(int year) {
  if(year%400 == 0) return true;
  else if(year%100 == 0) return false;
  else if(year%4 == 0) return true;
  else return false;
}
\end{verbatim}

\begin{verbatim}
bool isPrime(int num) {
  int divisor = 2;
  if(num <= 1) return false;
  // expressions can be evaluated within loop tests
  while(num/divisor >= divisor) {
    if(num % divisor == 0) {
      return false;
      divisor++
    }
  
  return true;
}
\end{verbatim}

\section*{Asserts}
\begin{verbatim}
// include assert library
#include<assert.h>

bool leap(int year) {
  // if year <= 1582, terminates with error
  assert(year > 1582);
  
  ...
}
\end{verbatim}

\section*{Seperate Compilation}
If you have multiple files (ie functions in one file, main program in another), declare the function in the main file as \begin{verbatim}void func(int number);\end{verbatim} and compile as \begin{verbatim}% gcc -o output functions.c main.c\end{verbatim}

\subsection*{Header Files}
You can also declare the functions in a header file as
\begin{verbatim}
#ifndef HEADER_H
#define HEADER_H
void func(int number);
#endif
\end{verbatim}
and in your main file \begin{verbatim}#include <header.h>\end{verbatim}

\section*{Recursion}
\begin{verbatim}
int gcd(int a, int b) {
  if(!b) return a;
  // call itself with updated/new arguments
  else return gcd(b,a%b);
}
\end{verbatim}

\section*{Locality}
\begin{verbatim}
void func(int a) {
  // changes the value of a... within func
  a = 42;
}

int main() {
  int a = 3;
  func(a);
  // a has not been changed in this scope
  printf("%d\n",a);
  
  return 0;
}
\end{verbatim}

\section*{Arrays}
\begin{verbatim}int a[3] = {10,30,50};\end{verbatim} creates an array, we can access the elements by \begin{verbatim}a[2];\end{verbatim} which returns 50. Very useful is
\begin{verbatim}
// beginning with i = 0, iterate n times
// n is the number of elements in a
// i is a count of how many times we've gone through the loop
for(int i = 0; i < sizeof(a)/sizeof(a[0]); i++) {
  // add the current element to the sum
  sum += a[i];
}
\end{verbatim}

Note that a for loop is defined as
\begin{verbatim}
for(initialization; condition; increment) {
  statement
}
\end{verbatim}
where any one or more parameters may be removed.

\section*{Floating-Point Numbers}
\begin{verbatim}
double x = 4/5;         // 0.0
double x = 4.0/5;       // 0.8
double x = (double)4/5; // 0.8
\end{verbatim}

\subsection*{Polynomials}
Polynomials are often represented as arrays, ie $3 + 4x - x^2$ is represented as \begin{verbatim}double f[] = {3,4,-1};\end{verbatim} and must be evaluated with \begin{verbatim}double eval(double f, int n, double x);\end{verbatim}

\subsubsection*{Example (Horner's Method)}
\begin{verbatim}
#include<stdio.h>

double horner(double f[], int n, double x) {
  double h = f[n-1];

  // declaring a variable within a loop statement requires compilation with c99 standards
  for(int i = n-2; i >= 0; i--) {
    h = h * x + f[i];
  }
  return h;
}

int main() {
  double f[] = {2,9,4,3};

  // we could replace the 4 by the sizeof() stuff we did earlier
  printf("%g\n", horner (f,4,0));
  printf("%g\n", horner (f,4,1));
  printf("%g\n", horner (f,4,2));

  return 0;
}
\end{verbatim}

\section*{Math}
\begin{verbatim}#include <math.h>\end{verbatim} has stuff like sin, cos, tan, exp, log, M\_PI (ie pi as a constant)...

\begin{verbatim}
#include<stdio.h>
#include<math.h>

// let's find the root of this function (if it's continuous on [a,b])
double f(double x) {
  return x-cos(x);
}

double bisect(double a, double b, double epsilon, int maxIters) {
  double m;

  for(int i = 0; i < maxIters; i++) {
    // find the mindpoint
    m = (a+b)/2;
    // fabs is from the math library, it return the absolute value of a variable
    if(fabs(f(m)) <= epsilon) return m;
    // figure out which half has the answer within it
    if(f(m) > 0) {
      b = m;
    }
    else  {
      a = m;
    }
  }
  return m;
}

int main() {
  printf("%g\n", bisect(-10,10,0.001,1000000);
  
  return 0;
}
\end{verbatim}

\section*{Structs}
\begin{verbatim}
// defines a variable containing two other variables
struct tod {
  int hours;
  int minutes;
};

int main() {
  // declare a struct like this
  struct tod now = {13,46};
  struct tod later;
  // access member variables
  later.hours = now.hours + 3;
  later.minutes = now.minutes;
  
  return 0;
}
\end{verbatim}
Structs can be passed as parameters or returned, just like pre-defined variables.

\subsection*{Typedefs}
\begin{verbatim}
// define a new type (not a struct anymore...)
typedef struct {
  int hours;
  int minutes;
} tod;

int main() {
  // notice the lack of struct in the definition?
  tod now = {13,53};
  
  return 0;
}
\end{verbatim}

\section*{Designated Initializers}
\begin{verbatim}int a[4] = {[2]=3, [0]=99};\end{verbatim} gives us $\{99,0,3,0\}$

\section*{Pointers}
\begin{verbatim}
int main() {
  // create a new integer in memory, assign it a value of 6
  int i = 6;
  // create a pointer to a memory address, have it point to the address of i
  int *p = &i;
  // set the contents pointed to by p to 10
  *p = 10;
  
  // prints 10
  printf("%d\n",i);
  
  return 0;
}
\end{verbatim}

\subsection*{Coding Format}
\begin{verbatim}int* p, i;\end{verbatim} is the same thing as \begin{verbatim}int *p, i;\end{verbatim} so \begin{verbatim}int *p;\end{verbatim} is prefered.

\subsection*{Assignment}
\begin{verbatim}
int main() {
  int i = 6;
  int *p = &i;
  int *q;
  // no stars here because we're assigning the memory address
  q = p;
  *q = 17;
  
  // prints 17
  printf("%d\n",i);
  
  return 0;
}
\end{verbatim}

\subsection*{Arguments}
\begin{verbatim}
void swap(int *p, int *q) {
  int temp = *p;
  // stars here so we assign the data
  *p = *q;
  *q = temp;
}

int main() {
  int i=1, j=2;
  swap (&i, &j);
  
  // prints 2, 1
  printf("%d, %d\n",i,j);
}
\end{verbatim}

\subsection*{Return Values}
\begin{verbatim}
int *largest(int a[], int n) {
  int m = 0;
  for(int i = 1; i < n; i++) {
    if(a[i] > a[m]) m = i;
  }
  // return the address as a pointer
  return &(a[m]);

int main() {
  int test[] = {0,1,2,3,2,1,0};
  // p will point to the largest element
  int *p = largest(test, sizeof(test)/sizeof(test[0]);
  *p = 100;
  
  // the third element is the largest... note how it has been changed to 100
  printf("%d\n",test[3]);

  return 0;
}
\end{verbatim}

\subsection*{Arithmetic}
\begin{verbatim}
int a[8], *p, *q. i;
p = &a[2];
q = p+3;   // q = &a[5]
p += 3;    // p = q = &a[5]
p = q - 3; / /p = &a[2]
i = q - p; // i = 3
\end{verbatim}

\subsection*{Memory}
\begin{verbatim}int *p = NULL;\end{verbatim} points to nothing (null pointer) and takes no memory. We can create dynamic storage and memory allocation with
\begin{verbatim}
#include <stdlib.h>

void *malloc(size_t size);           // allocates size amount of memory
void *realloc(void *p, size_t size); // reallocates previously allocated memory block
void free(void *p);                  // released memory allocated by malloc

// creates an array of size n filled with natural numbers one through n
int *numbers(int n) {
  int *p = (int *) malloc(n*sizeof(int)); // (int *) is technically optional, but good practice
  for (int i = 0; i < n; i++) p[i] = i+1;
  return p;
}
\end{verbatim}

\subsubsection*{Safety}
\begin{verbatim}
// adds an assert for error handling
void *safeMalloc(size_t size) {
  void *p = malloc(size);
  assert(p);
  return p;
}
\end{verbatim}

\subsection*{Structs}
\begin{verbatim}
struct tod{int hour, minute;};
struct tod *t = (struct tod *) malloc(sizeof(struct tod));
\end{verbatim}

\section*{Strings}
A string is an array of chars terminated by \begin{verbatim}"\0"\end{verbatim}

Examples:
\begin{verbatim}
char s[ ] = "hello";
char s[ ] = {'h', 'e', 'l', 'l', 'o', '\0'};
char *s = "hello";
\end{verbatim}

\begin{verbatim}
#include<stdio.h>

// first counter
int count(char s[ ], char c) {
  int n = 0;
  for(int i = 0; s[i] != '\0'; i++) {
    if(s[i] == c) n++;
  }
  return n;
}

// second alternative
int count (char *s, char c) {
  int n = 0;
  for (; *s; s++) {
    if(*s == c) n++;
  }
  return n;
}

int main() {
  char *hi = "Hello World!";
  printf("%d\n", count(hi,'l');

  return 0;
}
\end{verbatim}

\subsection*{String Library}
\begin{verbatim}
// string library
#include <string.h>

size_t strlen(const char *s);              // returns the length of the string
char *strncpy(char *s0, const char *s1);   // copies s1 onto s0 (s0 must be big enough
                                           // strncpy copies first n characters
char *strcat(char *s0, const char *s1);    // concatentation
int strcmp(const char *s0, const char *s1) // alpha betical compare gives <0, 0, or >0
\end{verbatim}

\section*{Vectors}
Vectors are better arrays with built in safety.

\begin{verbatim}
struct vector {
  int *a;           // actual array of data
  int size, length; // allocated and used storage
}

// proper initialization, for any new vector
struct vector *vectorCreate() {
  struct vector *v = (struct vector *) safeMalloc(sizeof(struct vector));
  v->size = 1;
  v->length = 0;
  v->a = (int *)safeMalloc(sizeof(int));
  return v;
}

// garbage handling and cleanup
struct vector *vectorDelete(struct vector *v) {
  // unnecessary, but recommended
  if(v) {
    free(v->a);
    free(v);
  }
  return (struct vector *) 0;
}

// behind-the-scenes assignment
void vectorSet(struct vector *, int index, int value) {
  assert(v && index>=0);
  if (index >= v->size) {
    do
      v->size *= 2;
    while(index >= v->size);
    v->a = (int *) safeRealloc((void *) v->a, v->size*sizeof(int));
  }
  while(index >= v->length) {
    v->a[v->length] = 0;
    v->length++;
  }
  v->a[index] = value;
}

// behind-the-scenes return
int VectorGet(struct vector *v, int index) {
  assert(v && index>=0 && index < v->length);
  return v->a[index];
}

// safe size function
int vectorLength(struct vector *v) {
  assert(v);
  return v->length;
}
\end{verbatim}

\section*{Big O Notation}
$O(n)$ is a measure of complexity. Examples: $3x^2 + 2 = O(x^2), 6\sin(x) = O(1), 13\log x = O(\log x)$. We can also have best and worst case complexity, ie best case: $O(1)$ (constant time), worst case: $O(n)$ (linear time). We also have logarithmic, quadratic, and polynomial time.

\section*{Sorting}
Different types of sorting, all have different time and memory complexities.

\subsection*{Selection}
Find the smallest element, swap it with the first element, repeat. Best case: $O(n^2)$, average case: $O(n^2)$, worst case: $O(n^2)$.

\subsection*{Insertion}
Find where an element should go, shift up elements above that, insert it, repeat n-1 times. Best case: $O(n)$, average case: $O(n^2)$, worst case: $O(n^2)$.

\subsection*{Merge}
Divide array in half, sort each half, merge the results. Sort results while merging. Sort "left side" of each array. Best case: $O(n \log n)$, average case: $O(n \log n)$, worst case: $O(n \log n)$.

\subsection*{Quick}
Pick a random pivot element, recursively sort both sides. Note this does not require a temporary array. It can also be improved by better choice of pivots, quicker partitioning methods, or optimized compiler functions. Best case: $O(n \log n)$, average case: $O(n \log n)$, worst case: $O(n^2)$.

\section*{Binary Search}
To search a sorted array, check the middle element, check the middle in whatever the direction your answer is, repeat. Time complexity is $O(\log n)$, which beats linear search ($O(n)$).



\end{document}
