\section{Chapter 18 -- Life in the Universe}
\begin{itemize}
\item Reasons why life likely might exist elsewhere
\begin{itemize}
\item Life arose quite quickly on Earth, so why not on other planets too
\item Chemicals on young Earth combined readily into complex organic compounds.  This may also be true on other exoplanets.
\item We have discovered microorganisms that could survive on other planets in our solar system
\end{itemize}
\item Timeline of development of life on Earth:\\
    \begin{tabular}{|c|c|}
    \hline
    Years ago & Event \\ \hline
    4.5 billion & Earth and moon form\\
    3.85 billion & Carbon isotope evidence of life\\
    3.5 billion & Oldest microfossil evidence of life\\
    2.5 billion & earliest evidence of oxygen in the atmosphere\\
    410 million & animals colonize land\\
    230 million & mammals and dinosaurs appear \\ \hline
    \end{tabular}
\item Requirements for life (on Earth)
\begin{itemize}
\item A source of nutrients
\item Energy to fuel the activities of life
\item Liquid water (* this is the only one that is difficult to achieve)
\end{itemize}
\item Microbes living in extreme conditions (volcanic vents, deserts) imply that if not for a need for water life could exist almost anywhere
\item Only likely candidates to have liquid water in our solar system are Mars and some jovian moons (e.g. Europa)
\item What properties must a star system have to contain life
\begin{itemize}
\item It must be older than several million years (that's how long life took to form after Earth's formation)
\item The star must not be much bigger than our Sun, because it would die off before life formed (still leaves about 99\% of stars)
\item Planets must have stable orbits (far less likely in binary star systems, but not impossible)
\item Bigger star means larger habitable zone (the zone where liquid water could exist)
\end{itemize}
\item A planet's spectra can give us the atmospheric makeup, allows us to look for water vapour, ozone, methane, etc.
\item Rare-Earth hypothesis
\begin{itemize}
\item The galaxy has a habitable zone, just like our solar system
\item Star systems further out contain far less non-hydrogen/helium elements, which almost completely compose terrestrial planets
\item Inner systems are subject to far more high supernovae, which would likely irradiate life
\item Leaves only about 10\% of solar systems habitable
\item Impact rate in our solar system dropped off quickly, is the same true everywhere?  In our solar system this was due to jovian planets ejecting small objects from the inner solar system
\item Our atmosphere has been relatively stable due to plate tectonics (which might be rare on other planets) regulating the carbon dioxide, and our large moon regulates our axial tilt
\end{itemize}
\item Counter-arguments include
\begin{itemize}
\item Earth is very small and wouldn't need a high abundance of heavy elements to form (relative to the mass of the star)
\item We don't know if a supernova would be harmful to life
\item If the Earth rotated faster it would also regulate our axial tilt without the Moon
\item Large moons could exist other places, ours isn't that rare
\item Life could adapt to a changing axial tilt
\end{itemize}
\item Drake equation: Number of civilizations = $N_\text{HP} * f_\text{life} * f_\text{civ} * f_\text{now}$
\begin{description}
\item[$N_\text{HP}$] number of habitable planets in the galaxy that \emph{could} have life
\item[$f_\text{life}$] fraction of habitable planets that actually have life
\item[$f_\text{civ}$] fraction of life-bearing planets upon which a civilization capable of interstellar communication \emph{has at some time} arisen
\item[$f_\text{now}$] fraction of civilization-bearing planets that currently have such a civilization
\end{description}
\item Fermi paradox: If it's so likely that other civilizations exist than there are some millions of years ahead of us technologically.  So where are they?  Options: ``we're alone'', ``every other civilization destroyed itself before it could settle the galaxy'' or ``they haven't revealed themselves to us yet''
\end{itemize}
