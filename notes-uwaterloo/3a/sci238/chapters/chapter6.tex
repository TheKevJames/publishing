\section{Chapter 6 -- Formation of Planetary Solar Systems}
\subsection{Overview}
This chapter is about the nature of our solar system and current scientific ideas about its birth. We will also examine characteristics about our solar system that are key clues about how it formed. Finally it will talk about how astronomers have discovered other planets around other stars and how those planetary systems are helping us understand our own.

\subsection{Clues to the formation of the Solar System}
The Solar System's layout and composition offer 4 clues as to how it was born.
\begin{enumerate}
\item {\bf Large bodies in the solar system have orderly motions}.
\begin{itemize}
\item All planets have nearly circular orbits going in the same direction in nearly the same plane.
\item  All planets {\bf orbit} the Sun in the same direction: {\bf counterclockwise} as viewed from high above Earth’s North Pole.
\item Most planets rotate in the same direction in which they orbit, with fairly small axis tilts. The Sun also rotates in this direction.
\item Most of the solar system’s large moons exhibit similar properties in their orbits around their planets, such as orbiting in their plane
\end{itemize}
\item {\bf Planets fall into two major categories}
\begin{enumerate}
\item {\bf Terrestrial planets}
\begin{itemize}
\item Small in mass and size
\item Close to the sun
\item Made of metal and rock
\item Few moons and no rings
\end{itemize}
\item {\bf Jovian planets}
\begin{itemize}
\item Large in mass and size
\item Far from the Sun
\item Made from H, He and hydrogen compounds
\item rings and many moons
\end{itemize}
\end{enumerate}
\item {\bf Swarms of asteroids and comets populate the solar system}. Vast numbers of rocky asteroids and icy comets are found throughout the solar system, but are concentrated in 3 distinct regions
\begin{itemize}
\item Most {\bf asteroids} orbit in the {\bf asteroid belt} between Mars and Jupiter
\item Many {\bf comets} are found in the {\bf Kuiper Belt} beyond Neptune's orbit
\item Even more {\bf comets} orbit the sun in the distant spherical region called the {\bf Oort Cloud} and only a rare few ever plunge into the inner solar system. May contain a trillion comets.
\end{itemize}
\item {\bf Several notable exceptions to these trends stand out}
\begin{itemize}
\item {\bf Uranus's odd tilt} Uranus rotates nearly on its side compared to its orbit and its rings and major moons share this sideways orientation.
\item {\bf Earth's relatively large moon} Our moon is much larger in size than most other moons in comparison to their planets
\item {Venus' backwards rotation}
\end{itemize}
\end{enumerate}

\subsection{The Sun}
The {\bf sun contains more than 99.8\% of the solar systems total mass}. (1000 times more massive than the rest of the solar system combined). The surface is a sea of rolling hot (5800K) Hydrogen and Helium gas. The sun is gaseous throughout and temperture and pressure both increase with depth. In addition, charged particles moving outward ({\bf Solar wind}) help shape planetary magnetic fields and can influence planetary atmospheres.

\subsection{Mercury}
A desolate, cratered world with no active volcanoes, no wind, no rain, and no life. It is a world of both {\bf hot and cold extremes}. The combination of rotation and orbit gives Mercury days and nights that last about 3 Earth months each. Mercury’s {\bf surface is heavily cratered} But it also shows evidence of past geological activity, such as plains created
by ancient lava flows and tall, steep cliffs that run hundreds of kilometers in length. Mercury’s {\bf high density} indicates that it has a very large iron core.

\subsection{Venus}
Venus is nearly identical in size to Earth. It {\bf rotates on its axis very slowly and in the opposite direction of Earth}, so days and nights are very long. An {\bf extreme greenhouse effect} bakes Venus’s surface to an incredible 470°C. Day and night. Venus has {\bf mountains, valleys, and craters, and shows many signs of past or present volcanic activity}. But Venus also has geological features unlike any on Earth, and we see no evidence of Earth-like plate tectonics.

\subsection{Earth}
Only planet in our solar system with oxygen to breathe, ozone to shield the surface from deadly solar radiation, and abundant surface water to nurture life. Temperatures are pleasant because Earth’s atmosphere contains just enough carbon dioxide and water vapor to maintain a moderate greenhouse effect.

\subsection{Mars}
Mars has ancient volcanoes. The presence of dried-up riverbeds, rock-strewn floodplains, and minerals that form in water offers clear evidence that Mars had at least some warm and wet periods in the past. Major flows of liquid water probably ceased at least 3 billion years ago. More than a dozen spacecraft have flown past, orbited, or landed on Mars.

\subsection{Jupiter}
Most famous feature: long-lived storm called the Great Pluto. {\bf Made primarily of hydrogen and helium and has no solid surface}. Increasing gas pressure would crush us long before we ever reached its core. Jupiter reigns over dozens of moons and a thin set of rings (too faint to be seen in most photographs).

\subsection{Saturn}
Saturn is made mostly of hydrogen and helium and has no solid surface. The rings may look solid from a distance, but in reality they are made of countless small particles, each of which orbits Saturn. Titan, the only moon in the solar system with a thick atmosphere. Saturn and its moons are so far from the Sun that Titan’s surface temperature is a
frigid -180°C.

\subsection{Uranus}
It is made largely of hydrogen, helium, and hydrogen compounds such as water (H 2 O), ammonia (NH3), and methane (CH4). Uranus lacks a solid surface. The entire Uranus system—planet, rings, and moon orbits—is tipped on its side compared to the rest of the planets and it gives Uranus the most extreme seasonal variations of any planet in our solar system.

\subsection {Neptune}
Neptune looks nearly like a twin of Uranus, although it is more strikingly blue. Smaller than Uranus but more massive due to higher density (even though they have similar composition. Triton is the only large moon in the solar system that orbits its planet “backward”—that is, in a direction opposite to the direction in which Neptune rotates. This backward orbit makes it a near certainty that Triton once orbited the Sun independently before somehow being captured into Neptune’s orbit.

\subsection {Pluto (and other dwarf planets)}
Pluto is much smaller and less massive than any of the other planets, and its orbit is much more eccentric and inclined to the ecliptic plane. Its composition of ice and rock is also quite different from that of any of those planets, although it is virtually identical to that of many known comets. Pluto is not even the largest of Kuiper belt objects, Eris, is slightly larger than Pluto.

\subsection{Where did the solar system come from?}
The {\bf nebular theory} begins with the idea that our solar system was born from a cloud of gas, called the solar nebula, that collapsed under its own gravity. As the solar nebula shrank in size, three important processes altered its density, temperature, and shape, changing it from a large, diffuse (spread-out) cloud to a much smaller spinning disk: {\bf heating, spinning, flattening}. As it collapsed, it heated up, spun faster and flattened into a disk.

\subsubsection{The Formation of the planets}
In the center of the collapsing solar nebula, gravity drew together enough material to form the Sun. Because hydrogen and helium gas made up 98\% of the solar nebula’s mass and did not condense, the vast majority of the nebula remained gaseous at all times. However, other materials could condense wherever the temperature allowed.

The solid metal and rock in the inner solar system ultimately grew into the terrestrial planets we see today, but these planets ended up relatively small in size because rock and metal made up such a small amount of the material in the solar nebula. {\bf Accretion} The process in which small ``seeds'' grow into planets.

The leading model for jovian planet formation holds that these planets formed as gravity drew gas around ice-rich ``boulders'' much more massive than Earth. Their large masses had gravity strong enough to capture some of the hydrogen and
helium gas that made up the vast majority of the surrounding solar nebula. Ultimately, the jovian planets accreted so
much gas that they bore little resemblance to the icy seeds from which
they started.

The young Sun had a strong solar wind—strong enough to have swept huge quantities of gas out of the solar system, sealing the compositional fate of the planets.

Asteroids and comets are leftover and are less common than earlier in the solar systems life. Although impacts occasionally still occur, the vast majority of these collisions occurred in the first few hundred million years of our solar system’s history, during the period we call the {\bf heavy bombardment}. These impacts did more than just batter the planets. They also brought materials from other regions of the solar system. (Potentially bringing water to Earth)

\subsubsection{How do we explain the exceptions to rules?}
Our moon - The giant impact hypothesis holds that a Mars-size object hit Earth at a speed and angle that blasted Earth’s outer layers into space. Strong support for this comes from two features of the moon's composition.
\begin{enumerate}
\item The composition of the moon is similar to Earth's outer layers
\item The Moon has a much smaller proportion of easily vaporized ingredients (such as water) than Earth. This fact supports the hypothesis because the heat of the impact would have vaporized these ingredients.
\end{enumerate}

Other Exceptions
\begin{itemize}
\item Mercury’s surprisingly high density may be the result of a giant impact that blasted away its outer, lower-density layers.
\item Giant impacts could have also been responsible for tilting the axes of many planets (including Earth) and perhaps for tipping Uranus on its side
\item Venus’s slow and backward rotation could also be the result of a giant impact
\end{itemize}
Nebular Theory accounts for {\bf all FOUR} of the major features of the solar system.

\subsubsection{When did the Planet's form?}
The planets begin to form just over {\bf 4.5 billion years ago}

\subsubsection{Other Planetary Systems}
We can detect planets in other stars directly (pictures) or indirectly (measurements of a stars properties).

Indirect Methods:
\begin{itemize}
\item {\bf Astrometric Technique} - we make very precise measurements of stellar positions in the sky. If a star “wobbles” gradually around its average position (the center of mass), we must be observing the influence of unseen planets.
\item {\bf Doppler technique} searches for a star’s orbital movement around the center of mass by looking for changing Doppler shifts in a star’s spectrum. Used for the majority of planet discoveries to date
\item {\bf Transits and Eclipses} searching for slight changes in a star’s brightness that
occur when a planet passes in front of or behind it. The transit method can also be used to search simultaneously for planets around vast numbers of stars and to detect much smaller planets than is possible with the Doppler technique.
\end{itemize}

\subsubsection{Comparing Extrasolar planets}
\subsubsection*{Orbits}
\begin{itemize}
\item Most of the planets orbit very close to their host star
\item many of the orbits are elliptical instead of nearly circular like the orbits of planets in our own solar system
\end{itemize}

\subsubsection*{Masses}
Most of the known extrasolar planets are more massive than Jupiter, smallest is twice the mass of Earth
