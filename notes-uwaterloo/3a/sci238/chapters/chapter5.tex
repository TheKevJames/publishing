\section{Chapter 5 -- Light}
\begin{itemize}
\item light spectrum
\begin{itemize}
\item the range of visible light we see as we pass it through a prism
\item can be detailed enough to show us individual black bands representing missing colors
\end{itemize}
\item electromagnetic spectrum: the complete version of all light forms. visible light is only tiny fraction
\item wave properties
\begin{itemize}
\item a wave is something that can transmit energy without having to carry material with it
\item wavelength: the distance between adjacent peaks (measured in m)
\item frequency: the number of cycles per second (measured in Hz)
\item since light can affect both charged particles and magnets, we refer to it as an electromagnetic wave
\end{itemize}
\item All light travels at the same speed: $3*10^8 m/s$
\item particle properties
\begin{itemize}
\item comes in pieces called photons
\end{itemize}
\item gamma | x-ray | ultraviolet | visible | infrared | microwave | radio
\item atomic structure
\begin{itemize}
\item protons, neutrons, electrons
\item some really obvious shit
\item strong nuclear force holds protons and neutrons together at the nucleus
\item electrons in an atom a cloud that surrounds the nucleus and gives the atom its apparent size
\item it is impossible to pinpoint their positions
\item isotopes: atoms with varying numbers of neutrons
\end{itemize}
\item light and matter interact in four ways
\begin{enumerate}
\item emission: matter creates light (eg thermal, electrical potential)
\item absorption: light's energy is absorbed into matter (typically as thermal)
\item transmission: matter allowing light to pass through
\item reflection: light bouncing (scattering is reflecting randomly)
\end{enumerate}
\item transparent: material allows light to pass through
\item opaque: material absorbs all light
\item some materials treat certain light differently (eg red glass allows red light through, grass reflects green light)
\item types of spectrum
\begin{itemize}
\item continuous: broad range of wavelengths without interruption
\item emission line: light composed of individual bands (dependant on composition and temperature)
\item absorption line: broad range with individual bands missing
\end{itemize}
\item electrons can only have particular amount of energy and not levels inbetween (measured in eV)
\item 1 electron Volt = $1.60*10^{-19}$ Joule
\end{itemize}

These values are unique to every atom, ion, and molecule. Example:
\begin{verbatim}
A hydrogen atom has states:
    level 1: 0 eV   ground
          2: 10.2
          3: 12.1
          4: 12.8
 ionization: 13.6   escape
\end{verbatim}
% (((if possible, include figure 5.09 pg 119)))

\begin{itemize}
\item an electron rising or falling between levls are called energy level transitions
\item can only occur when an election gains or loses the specific amount of energy separating the two levels
\item electrons will not accept energy in quantities that do not correspond with an energy level (except escape, can go over)
\item each possible downward energy level transition (eg 3 to 1) corresponds to a release of energy in the form of photons
\item the more energy released, the shorter the photon wavelength
\item some of these transitions create photons with wavelengths that fall in the visible light spectrum (many won't be visible)
\item the exact possible emissions are unique to each atom, so we can use emission lines to identify compositions of light sources
\item electrons will not rest at high energy levels, they will fall back down quickly (fraction of a second)
% (((if possible, include figure 5.10 pg 120)))
\item energy exchanges typically caused by atoms colliding
\item most just bump away, a few transfer exactly the right amount of energy to an electron
\item collisions only keep happening while the gas is relatively hot
\item energy can also be recieved in these exact quantities by other photons
\item this causes absorption lines, since the photons that get absorbed then released will be in random directions
\item the photons that do not get absorbed travel straight through the material to us
\item most objects (eg rocks, plants, people) are complex enough that they absorb light from a very broad range
\item light cannot easily pass through, light emitted cannot easily escape
\end{itemize}

Consider an idealized case, an object absorbs all photons and does not allow photons inside it to escape easily. Photons bounce randomly around inside the object, constantly exchanging energy with its atoms or molecules. When photons finally escape the object, their radiative energies have become randomized so that they are spread over a wide range of wavelengths. This wide range explains why the spectrum of light from such an object is continuous, like a pure rainbow without any absorption or emission lines

\begin{itemize}
\item this all depends on the temperature of the object (the average kinetic energy of the atoms)
\item that's why this type of light is called thermal radiation
\item no real object emits a perfect thermal radiaton spectrum, but all objects emit light that approximates it
\end{itemize}
% (((if possible, include figure 5.11 pg 122)))

\begin{itemize}
\item two laws of thermal radiation:
\begin{enumerate}
\item each square meter of a hotter object’s surface emits more light at all wavelengths
\begin{itemize}
\item each square meter on a hotter star emits more light at every wavelength than a cooler star
\item hotter star emits light at some ultraviolet wavelengths that the cooler star does not emit at all
\item T = temperature (in Kelvin)
\item $o = 5.7*10^{-8} \frac{W}{m^2 K^4}$
\item emitted power per square meter of surface = $oT^4$
% (((if possible, include figure 5.12 pg 123)))
\end{itemize}
\item hotter objects emit photons with a higher average energy
\begin{itemize}
\item shorter average wavelength
\item explains why peak wavelength is shorter in hotter objects
\item Lambda Peak = $2.9*10^6 \frac{1}{T}$
\end{itemize}
\end{enumerate}
\item Doppler effect
\begin{itemize}
\item the velocity of the light source have an effect on the apparent wavelength of the photons
\item travelling away: wavelengths appear longer, become redshifted
\item travelling towards: wavelengths appear shorter, become blueshifted
\item v = velocity of source (in radial direction, so only net away from us. negative means towards)
\item c = speed of light ($3*10^8 m/s$)
\item Lambda Shift = observed wavelengths from source
\item Lambda Rest = lab recorded wavelengths of source's composition
\item v/c = (shift - rest) / rest
% (((if possible, include figures 5.13,14,15 pg 124,5)))
\end{itemize}
\item telescopes have two properties
\begin{enumerate}
\item light collecting area - total light telescope can collect at a time
\item angular resolution - smallest angle we can determine two stars are distinct
\end{enumerate}
\item arcminute: 1/60th of a degree
\item arcsecond: 1/60th of an arcminute
\item human eye has angular resolution of ~1 arc minute (objects se)
\item basic telescope designs
\begin{enumerate}
\item refracting telescopes
\begin{itemize}
\item operates like eye, uses transparent glass lenses to collect and focus light
\item earliest telescopes built by Galileo
\item worlds largest has 1m lens, 19.5m tube, completed 1897
% (((if possible, include figure 5.17 pg 128)))
\end{itemize}
\item reflecting telescopes
\begin{itemize}
\item use large curved primary mirrors to gather light, small secondary mirror to reflect to a focus
\item majority of modern telescopes, size typically restricted by weight of primary mirrors
\item 5m Hale telescope on Mt. Palomar, San Diego (1948)
\item 8m Gemini telescope on Mauna Kea, Hawaii
\item 10m Keck telescopes in Hawaii
\item plans for 30m telescope in 2018
\end{itemize}
% (((if possible, include figure 5.18 pg 129)))
\end{enumerate}
\end{itemize}

\begin{itemize}
\item Telescopes can be specially designed to recieve non-visible light
\begin{itemize}
\item radio waves have large wavelengths, telescopes need to be large to get decent resolution
\item 305m Arecibo radio dish in Puerto Rico has only ~1 arc minute
\item collecting high energy rays straight on will punch straight through the lens, possibly damaging it
\item need to catch them at a slight angle and deflect them into the focus
% (((if possible, include figure 5.21b pg 130)))
\end{itemize}
\item Putting telescopes in space dodges atmospheric interference but is extremely expensive
\begin{itemize}
\item bright day skys, light pollution, weather, air turbulence (causes eye-visible twinkling)
\item lower energy wavelengths are barely affected by atmosphere and can be built lower
\item most high energy wavelengths cant even reach the ground
% (((if possible, include figure 5.25 pg 133)))
\end{itemize}
\item adaptive optics: computers control the telescopes, making minute adjustments to the shape of the mirrors many times a second to eliminate atmospheric distortion
\item interferometry: using multiple telescopes simultaneously to combine their images to achieve higher resolution
\end{itemize}
