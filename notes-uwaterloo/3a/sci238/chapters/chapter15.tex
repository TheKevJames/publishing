\section{Chapter 15 -- Galaxies and the Foundation of Modern Cosmology}
By taking a picture of a small section of the sky and determining how many such pictures would be necessary to cover the entire sky, we can extrapolate that there are well over 100 billion galaxies within the observable universe. Each of these galaxies has a different shape, size, color, etc.

\subsection{Types of Galaxies}
We have three major categories of galaxies:
\begin{description}
\item[Spiral galaxies] are flat white disks with yellowish bulges near the center. The disks are fileed with cool gas and dust, as well as some sparse ionized gasses, and usually have a few spiral arms. The Milky Way is a spiral galaxy.
\item[Elliptic galaxies] are redder, rounder, and tend toward being longer than they are wide. They contain less cool gas and dust than spiral galaxies and more of the hot ionized gasses.
\item[Irregular galaxies] appear lke neither of these categories.
\end{description}

The reason the galaxies are different colors is based on the relative ratios of stars of different colors within them: spiral and irregular galaxies are white-ish since they contain stars of all colors and ages, while elliptic galaxies are reddish since they are populated mostly by old and red stars.

We also categorize galaxies by size: {\bf dwarf galaxies} contain as few as 100 million stars and {\bf giant galaxies} contain more than 1 trillion.

\subsubsection{Spiral Galaxies}
Spiral galaxies have a thin {\bf disk} which forms outwards from a central {\bf bulge}. The disk smoothly merges into a dim {\bf halo} with a radius which can be upwards of 100 thousand lightyears. The {\bf disk population [Population 1]} (population of stars within the disk) includes stars of all masses and ages. The {\bf spheroidal population [Population 2]} consists of halo and bulge stars, the halo stars being generally old and low in mass.

We thus define:
\begin{itemize}
\item The {\bf disk component} is the flat disk in which stars follow orderly, nearly circular orbits around the galactic center. The disk component always contains an interstellar medium of gas and dust, but the amounts and proportions of molecular, atomic, and ionized gases in this medium differ from one spiral galaxy to the next.
\item The bulge and halo together make up the {\bf spheroidal component}, named for its rounded shape. Stars in the spheroidal component have orbits with many different inclinations, and the spheroidal component generally contains little cool gas and dust.
\end{itemize}

All spiral galaxies have both of these components, though some {\bf barred spiral galaxies} have a straight bar of stars cutting through the center, with arms spiralling off of the ends of the bar. {\bf Lenticular galaxies} are somewhat of a halfway between spiral galaxies and elliptical galaxies, as they have no spiral arms.

Approximately 75-85\% of galaxies are spiral or lenticular.

\subsubsection{Elliptical Galaxies}
Elliptical galaxies have only a spherical component and no significant disk component. For this reason, they are sometimes called {\bf spheroidal galaxies}. These galaxies tend to be small (and small elliptical galaxies are the most common of galaxies), though some are the most massive of galaxies: {\bf giant elliptical galaxies}.

The composition of elliptical galaxies (ie. the ionized gasses) are much like the hot X-ray-producing gasses generated by supernovae and powerful stellar winds elsewhere in the universe.

Since these galaxies do not have many of the cool gasses found in other galaxies, they have very little ongoing star formation. This is why tese galaxies appear reddish: they tend to have very few young blue stars to counteract the color of the old red and yellow ones.

\subsubsection{Irregular Galaxies}
The irregular galaxies are all othre galaxies, which we can not easily classify. They are usually white and dusty and contain young massive stars. These also tend to be the oldest of galaxies: more irregular galaxies can be found the farther away we look. Though we aren't sure why, it seems that irregular galaxies were more common when the universe was younger.

\subsubsection{Hubble's Galaxy Classes}
Hubble designed a system for classifying galaxies: elliptical galaxies have a designation of E followed by a number from zero to seven, with a larger number signifying a larger eccenricity in shape: ie. an E0 galaxy is a sphere. Spiral and barred spiral galaxies have respective designations S and SB, followed by a lowercase letter from ``a'' to ``c'', where ``c'' corresponds to the smallest bulge and largest amount of dusty gas. Lenticular galaxies are designated S0 and irregulars are Irr.

\subsection{Measuring Distance}
We can measure the distance between the Earth and a galaxy using {\bf parallax}. To do this, we must know the distance between the Earth and the Sun. This is done using {\bf radar ranging}: by bouncing radio waves off of Venus and determining how long they take to return, we can find the distance to Venus. Keppler's laws, then, give us the distance to the Sun.

Since we can only measure within about a few hundred lightyears with parallax, we must also learn to measure distance by the inverse-square law of luminosity. Since similar stars should have similar luminosities (ie. a main-sequence G2 star like the Sun would have a similar luminosity), we can use the inverse-square law to determine how much farther it is from us than the Sun. For this approach, we must find {\bf standard candles}: objects whose luminosity is known by which we can compare against.

Sun-like stars do not make very good candles since they are somewhat dim. To measure distances beyond a thousand lightyears, we need brighter candles. We thus have an approach by which we can get progressively better estimates: find a star within parallax distance, plot its HR, and establish luminosity from distance and brightness; then measure brightness of stars too far for parallax and use the inverse-square law to determine approximate distance.

Since we tend to use main-sequence stars for this, we refer to this technique as {\bf main-sequence fitting}.

Unfortunately, this approach does not work well outside of our galaxy. We use brighter stars {\bf cepheid variable stars}, or {\bf cepheids}, for this task. These stars vary in brightness at some constant rate, from our perspective. The periods, though, are closely related to their luminosities: longer periods are found on more luminous stars. Cepheids, then, obey a {\bf period-luminosity relation} which allows us to estimate their luminosity within 10\% simply by measuring their period. A Cepheid with a period of 30 days is approximately ten thousand times brighter than the Sun.

Cepheids vary like this due to varying amounts of energy radiating from their surface: they have a peculiar problem in matching the amount of energy their surfaces radiate with the amount welling up from the core. The upper layers of a Cepheid variable star alternately expand and contract to attempt to find equilibrium, causing the star’s luminosity to rise and fall. The period–luminosity relation holds because larger (more luminous) Cepheids take longer to pulsate in size.

We can use Cepheids as a stepping stone to find even brighter distant standard candles.

Some of the best distant standard candles are white dwarf supernovae, which are believed to be white dwarfs which have expanded beyond 1.4 times the mass of the Sun. Since these have a similar mass, these should all have comparable luminosities. Their luminosity is approximately ten billion times that of our Sun, and so we can detect them even in galaxies billions of lightyears away. The major disadvantage to this approach, of course, is that we can only measure the distance to galaxies with a supernovae-ing white dwarf; and this only happens once every few hundred years in the average galaxy. This technique, though, does allow us to calibrate an even better technique based on the expansion of the universe.

The spectra of most spiral galaxies tends to be redshifted; which occurs when a radiating object is moving away from us. When we measure the distance (using the above methods) as well as the redshifts of various galaxies, we notice that galaxies farther away from us are moving away at a faster rate. Thus, we determine that the universe is expanding. We express this with {\bf Hubble's Law} \[ v = H_0 \times d \] where $v$ is an object's velocity away from us and $H_0$ is Hubble's constant. Note that astronomer's tend to use this law in reverse: using a galaxy's speed to measure its distance away from us.

Unfortunately, this is only an approximation, as the speed of a galaxy is impacted by the effects of gravity from nearby galaxies as well as from the expansion of the universe. In addition, we base our approximations upon how closely we can approximate Hubble's constant ($H_0 = 22 \frac{km}{Mly \times s}$).

Note that the first of these issues impacts us most when measuring distances within the local group, as these are attracted to us by the Milky Way and thus move away from us at a much smaller rate than expansion would imply.

The major problem with measuring distances to galaxies is this chain of measurements; even today, based on the uncertainty at each step we can only be confident as to a galaxy's distance within about ten percent.

Remember, this chain is:
\begin{enumerate}
\item Radar Ranging
\item Parallax
\item Main-sequence Fitting
\item Cepheid Variables
\item White Dwarf Supernovae
\item Hubble's Law
\end{enumerate}

\subsection{Age of the Universe}
All our observations are consistent with the {\bf Cosmological Principle}: that the universe appears identical at all locations. In other words, it has no ``edge'' or ``center''. More specifically, the universe is expanding -- but it is not expanding \emph{into} anything, nor is it expanding into nothing. It itself is an infinite, three-dimensional surface which has no edges, sides, or center.

The Hubble Constant, then, changes as the universe ages: at any given time $\frac{1}{H_0}$ is exactly equal to the age of the universe. Technically, the Hubble Constant is non-constant, then, but it varies slowly enough as to be virtually constant.

Based on our current estimate of Hubble's Constant, the universe is between 12 and 15 billion years old. To be more precise, we would need to know whether the rate of expansion is accelerating, which could change these values immensely: if the expansion rate has been increasing, the age of the universe woulld be somewhat more than $\frac{1}{H_0}$, and vice-versa. Our current best-estimate is that the universe is 14 billion years old.

\subsubsection{Lookback Times}
Since the universe is expanding, it can be difficult to refer to the distances to objects. If we see light from an object which left that object 400 million lightyears ago, then it is currently more than 400 million lightyears away. An object's {\bf lookback time} is the difference between the current age of the universe and the age the universe was when light left that object. The lookback time of an object, then, is unambiguous.

The lookback time of an object is directly related to its redshift. This is because the expansion of our universe also stretches out the photons within it, thus giving us a {\bf cosmological redshift} as well as a Doppler redshift. This is a difference in perspective, mostly, as we can either think of galaxies as hurtling through space or being carried along by the expanding universe.

The {\bf cosmological horizon} represents the limits of the observable universe as a boundary in time, instead of space: in a universe 14 billion years old, we can not see any objects with lookback times greater than 24 billion years.

\subsection{Evolution of Galaxies}
We know far less about the life-cycles of galaxies than we do of stars. That said, we can use galaxies of various lookback times to view galaxies of different ages. We can not see far enough back to watch galaxies being formed, but we can determine their likely early life based on some assumptions:
\begin{itemize}
\item Hydrogen and Helium gas filled space uniformly soon after the birth of the universe
\item The distribution of matter in the early universe was not perfectly uniform
\end{itemize}

We assume the denser areas grew into galaxies based on our understanding of the laws of physics. These regions of enhanced density would have expanded along with the rest of the universe, gradually slowing their expansion due to ever-increasing effects of gravity. Within a billion years, their expansion would have reversed, the material within them forming {\bf protogalactic clouds}, which eventually formed galaxies.

The clouds which would eventually form spiral galaxies cooled as they contracted, and the first stars grew from the coldest, densest clumps of gas. These stars were likely massive, with lifespans of only a few million years. Their supernovae seeded these clouds with heavier elements and heated the surrounding gasses. This heating would have slowed the collapse of the clouds and their rate of star formation, allowing time for the gasses to form rotating disks.

This explains the shape of spiral galaxies: the spheroidal center consists of stars formed in the early stages, before a definite rotational plane was established, and thus have verying planes of rotation. Those formed on the arms were formed after a rotation had been established, and thus all follow the same plane.

This model, though, does not explain irregular and elliptical galaxies.

\subsubsection{Variances in Galaxies}
We attempt to determine why these galaxies differ by examining their differences: why do spiral galaxies have gas-rich disks, while other galaxies do not?

Two plausible explanations for the differences between spiral galaxies and elliptical galaxies trace a galaxy’s type back to the protogalactic cloud from which it formed:
\begin{description}
\item[Protogalactic Spin] A galaxy’s type might be determined by the spin of the protogalactic cloud from which it formed. If the original cloud had a significant amount of angular momentum, it would have rotated quickly as it collapsed. The galaxy it produced would therefore have tended to form a disk, and the resulting galaxy would be a spiral. If the protogalactic cloud had little or no angular momentum, its gas might not have formed a disk at all, and the resulting galaxy would be elliptical.
\item[Protogalactic Density] A galaxy’s type might be determined by the density of the protogalactic cloud from which it formed. A protogalactic cloud with relatively high gas density would have radiated energy more effectively and cooled more quickly, thereby allowing more rapid star formation. If the star formation proceeded fast enough, all the gas could have been turned into stars before any of it had time to settle into a disk, making it an elliptical galaxy. In contrast, a lower-density cloud would have formed stars more slowly, leaving plenty of gas to form the disk of a spiral galaxy.
\end{description}

The second theory is consistent with observations: young elliptical galaxies tend to have very few young stars, implying their stars were all formed very quickly and that new star formation is not ongoing for long.

Another possible avenue for determining why galaxies differ is by looking at what changes after they are formed. Galaxies are not formed in isolation, and their interactions with other galaxies may be the cause of the differences.

Sometimes, galaxies may collide. These are immense interstellar events which cause enormous changes to the objects involved. These collisions were much more common in the early universe -- back when galaxies were much closer together.

Based on computer simulations, we see that the collision of two spiral galaxies can form an elliptical galaxy since tremendous tidal forces rip the disks apart and a large fraction of the gasses sink to the center of teh collision and rapidly form new stars. Little of the disks remain in the end, and the stars have randomized orbits.

Elliptical galaxies are most common in areas of the universe with a large number of galaxies -- which would be the case if they were often formed by the collision of other galaxies. Our observations tend to lend creedence to elliptical galaxies being formed this way. Elliptical galaxies tend to have structures corresponding to likely violent pasts and by observing {\bf central dominant galaxies} we see that elliptical galaxies can grow to a large size by consuming other galaxies through {\bf galactic cannibalism}.

Galactic collisions could also ignite huge bouts of star formation -- {\bf starbursts} -- which can form entire {\bf starburst galaxies}. Since these would consume al ltheir gasses extremely quickly, they would rapidly burst and emit {\bf galactic winds} which carry away all gasses capable of supporting the constant star formation of a spiral galaxy.

\subsection{Quasars and Other Active Galactic Nuclei}
Some stars have extreme amounts of radiation and jets of material from their cores. These very crazy cores are called \textbf{active galactic nulcei}, the most luminous of these are called \textbf{quasars}. Quasars are only found at great distances wich tell us that they were much more common billions of years ago, from this we infer that quasar production decreases as galaxies age.

\subsubsection{Quasars}
The current theory is that the energy in a quasar comes from the accretion disk around supermassive black holes. Quasars were discovered when a scientist was mapping radio sources with visible objects and found a blue star that have emission lines that didnt appear to belong to any known chemical element. It was eventually found that these emission lines were just those of hydrogen that had been hugley redshifted. From there the objects distance and luminosity were calculated and shit was bright.

While quasars only appear very far away, we can find active galactic nuclei closer to home. Unfortunatly these suckers are small (only about 100 light years across) so it is very hard to resolve them. Using interferometry with radio images we have found that they are even smaller (less than 3 light years across) and the way they flicker implies they are even smaller.

Certain galaxies also emit unusually strong radio waves, called \textbf{radio galaxies}. These waves come from huge radio lobes on either side of the galaxy. At the center of the galaxy is a active galactic nuclei with two gigantic jets of plasma shooting into the radio lobes. Recent discoveries imply that radio galaxies and quasars are actually the same thing veiwed in different ways.

\subsubsection{Power Source of Quasars and Active Galactic Nuclei}
Currently we think that the energy of quasars and AGN comes from matter falling into black holes. The matter falling converts gravitational potential into kinetic energy and matter colliding on the way down converts that to thermal energy, and that resulting heat emits the crazy radiation we see.

This explains their crazy luminosities, how they emit radiation over a broad range of wavelengths, and their jets. Accretion disks convert 10-40\% of mass into energy (much greater than a stars 1\% conversion) which explains the high luminosities. Hot gas near the accretion disk emits ultraviolet and X-ray photons. This radiation ionizes near by interstellar gas which emits visible light (and the emission lines that led to their discovery). Dust grains in surrounding molecular clouds absorbe this light and emit infrared ratiation. The fast electrons in the jet emit radio radiation. The prescense of jets is harder to explain. We think its related to twisted magnetic fields caused by the spinning of the accretion disks.

\subsubsection{Supermassive Black Holes}
Some astronomers doubt the existence of supermassive black holes, and finding them is very difficult. By observing matter orbiting the centers of nearby galaxies we find that supermassive black holes are very common and possibly at the center of every galaxy. We do this by looking at the doppler shifts on either side of where we think the black hole is. If it is red on one side and blue on the other it means that gas is orbiting some unseen object. We can use mass and distance calculations to get the mass of the object at the center of this orbit. Many objects (molecular clouds in particular) orbit very close to black holes (less than 1 light year), we can use this to guess the volume of it.

Black hole like objects appear in the center of a wide variety of galaxies with grossly different properties which implies that they are important to the formation of a galaxy, we just dont know how yet.
