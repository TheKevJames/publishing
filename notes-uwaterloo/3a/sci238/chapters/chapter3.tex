\section{Chapter 3 -- Ancient Astronomy}
\subsection{Ancient Roots of Science}
Other lunar calendars remain roughly synchronized with solar calendars by taking advantage of an interesting coincidence: 19 years on a solar calendar is almost precisely 235 months on a lunar calendar. As a result, the lunar phases repeat on the same dates about every 19 years (a pattern known as the Metonic cycle).

The study of ancient structures in search of astronomical connections is called {\bf archaeoastronomy}

\subsection{Ancient Greek Science}
Greek - To account for the fact that the Sun and Moon each move gradually eastward through the constellations, the Greeks added separate spheres for them, with these spheres turning at different rates from the sphere of the stars. The planets also move relative to the stars, so the Greeks added additional spheres for each planet. The {\bf difficulty} with this model was that it made it hard to explain the apparent retrograde motion of the planets.

The {\bf Ptolemaic model} - Each planet moves on a small circle whose center moves around the Earth. A planet following this circle-upon-circle motion traces a loop as seen from Earth, with the backward portion of the loop mimicking apparent retrograde motion. Despite its complexity, the Ptolemaic model proved remarkably successful: It could correctly forecast future planetary positions to within a few degrees of arc.

\subsection{The Copernican Revolution}
Copernicus - Discovered simple geometric relationships that allowed him to calculate each planet’s orbital period around the Sun and its relative distance from the Sun in terms of Earth–Sun distance. The model’s success in providing a geometric layout for the solar system further convinced him that the Sun-centered idea must be correct.

Kepler’s key discovery was that planetary orbits are not circles but instead are a special type of oval called an ellipse. The locations of the two tacks are called the foci (singular, focus) of the ellipse. The long axis of the ellipse is called its major axis, each half is called the semimajor axis. A circle is an ellipse with zero
eccentricity, and greater eccentricity means a more elongated ellipse.

{\bf Kepler's first law} - The orbit of each planet about the sun is an elipse with the sun at one focus. It is closest at {\bf perihelion} and farthest at the point called the {\bf aphelion}.

{\bf Kepler's Second Law} - As a planets moves around its orbit it sweeps out equal areas in equal times. This
means the planet moves a greater distance when it is near perihelion than it does in the same amount of time near aphelion. That is, the planet travels faster when it is nearer to the Sun and slower when it is farther from the Sun.

{\bf Kepler's Third Law} - More distant planets orbit the sun at slower average speeds. {\bf $p^2 = a^3$} p is the orbital period in years, and a is the distance from the sun. We can use the law to calculate a planet’s average orbital speed.

Galileo - observed four moons clearly orbiting Jupiter, not Earth. Soon thereafter, he observed that Venus goes through phases in a way that proved that it must orbit the Sun and not Earth.

Galileo’s experiments and telescopic observations overcame remaining objections to the Copernican idea of Earth
Venus as a planet orbiting the Sun. Although not everyone accepted his results immediately, in hindsight we see that
Galileo sealed the case for the Sun-centered solar system.

Science generally exhibits three hallmarks: (1) Modern science seeks explanations for observed phenomena that rely solely on natural causes. (2) Science progresses through the creation and testing of models of nature that explain the observations as simply as possible. (3) A scientific model must make testable predictions about natural phenomena that would force us to revise or abandon the model if the predictions do not agree with observations.

A {\bf scientific theory} is a simple yet powerful model that explains a wide variety of observations using just a few
general principles, and that has survived repeated and varied testing.
