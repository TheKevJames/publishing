\section{Chapter 4 -- Motion, Energy, and Gravity}
\begin{description}
\item[velocity] speed with respect to a direction
\item[acceleration] rate of change in velocity
\item[the acceleration of gravity] $g = ~9.8 \frac{m}{s^2}$
\item[momentum] $p = mv$
\item[net force] net force acting on an object dictates changes in its momentum
\item[mass] amount of matter in object
\item[weight] force that a mass exerts on the ground (due to gravity and other forces, like elevators)
\item[freefall] state of falling without any resistance to slow you down. Since there is no ground for your mass to exert force upon, you seem weightless. Object $x$ in orbit of object $y$ is in a constant state of freefall toward $y$
\item[Newton's Laws]
\begin{enumerate}
\item If net force is zero, objects maintain velocity
\item Force is mass times acceleration
\item Forces have equal and opposite reaction forces
\end{enumerate}
\end{description}

\begin{itemize}
\item Conservation of Angular Momentum
\begin{itemize}
\item total angular momentum can never change
\item an object's angular momentum must transfer to or from another object
\item angular momentum of an object = radius of the orbit * velocity around the orbit * mass
\item this explains
\begin{enumerate}
\item why Earth needs no fuel or push to continue orbiting the sun
\item why Earth's velocity around the sun must be faster when it is closer to the sun (Kepler's Law)
\end{enumerate}
\item The same is true for Earth rotating on its axis, though it is slowly transerring this rotational angular momentum to the Moon
\end{itemize}
\item Conservation of Energy
\begin{itemize}
\item energy also cannot change, only transferred and converted
\begin{itemize}
\item kinetic: movement
\item radiative: carried by light (visible or otherwise)
\item potential: stored in some form (eg gravitational potential)
\item thermal: subcategory of kinetic, sum movement of particles within (measured in Kelvin)
\item mass: a type of potential energy  (energy = mass * speed of light$^2$)
\end{itemize}
\end{itemize}
\item Universal Law of Gravitation
\begin{itemize}
\item every mass attracts every other mass in the universe
\item $G = 6.67*10^{-11} \frac{m^3}{kg*s^2}$
\item d = distance between object 1 and 2
\item $F = \frac{G*m_1*m_2}{d^2}$
\end{itemize}
\end{itemize}

\begin{description}
\item[bound orbits] an object goes around another object over and over again (planets, satellites, moons)
\item[unbound orbits] paths that bring an object close to another object just once (some outer comets)
\end{description}

% (((if possible, include figure 4.15 pg 99)))

\begin{itemize}
\item Newton's Version of Kepler's Third Law
\begin{itemize}
\item orginally $p^2 = a^3$
\item a = average orbital radius
\item p = orbital period
\item Newton: $p^2 = a^3 * \frac{4\pi^2}{G(m_1 + m_2)}$
\item note: if m1 >> m2, you may be able to ignore m2 for an approximate answer (easier expression manipulation)
\end{itemize}
\end{itemize}

\begin{description}
\item[Orbital Energy] the sum of an object's gravitational potential and kinetic energy. this value remains constant
\item[Gravitational Encounter] two objects pass near enough so that each can feel the effects of the other’s gravity. Note: gravitational encounters are the basis for slingshot maneuvers in spacecraft
\end{description}

\begin{itemize}
\item Atmospheric Drag
\begin{itemize}
\item low-orbiting objects (few hundred km) experience slight drag
\item causing it to slow, converting to thermal enegry
\item as it slows its orbit becomes smaller, eventually crashing to the surface
\end{itemize}
\item Escape Velocity
\begin{itemize}
\item if an object in bound orbit gains velocity (eg thrust) its orbit increases average altitude
\item an object with velocity of 11000 m/s (on surface) will escape Earth's gravity
\item this velocity does not depend on mass
\item since gravity weakens with distance, starting higher requires less energy
\end{itemize}
\item Tides
\begin{itemize}
\item the moon can pull the water on the surface towards it
\item strongest on the side of Earth facing the moon, causing a bulge
\item weakest on far side of Earth, causing another bulge
\item Sun also has an effect on tides, but is too far away to be as noticeable (little under half of moon's)
\item spring tides (sun and moon work together) and neap tides (sun and moon fight)
% (((if possible, include figure 4.21 pg 104)))
\item moon's synchronous rotation caused by tidal friction
\item used to rotate faster, tidal bulges in its mass
\item tidal forces cause the axial rotation to slow (and orbit radii to increase)
% (((if possible, include figure 2 pg 103)))
\end{itemize}
\end{itemize}
