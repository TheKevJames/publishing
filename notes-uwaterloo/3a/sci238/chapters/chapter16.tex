\section{Chapter 16 -- Dark Matter, Dark Energy, and the Fate of the Universe}
The dominant source of gravity in the universe is {\bf dark matter}, which is completely unobservable. {\bf Dark energy} seems to be counteracting the effects of gravity on a massive scale.

In fact, the universe itself seems to be mostly composed of dark matter, rather than of atoms. Basically, dark matter is a theorized bunch of matter than may or may not exist but is necessary to create the effects our models predict. It is matter that gives off no light, ie. remains ``dark''.

We predict that the expansion of the universe must slow over time due to the diminishing effects of gravity. If it does not, there must exist some ``dark'' energy fueling the expansion. Note that sometimes we refer to dark energy as {\bf quintessence} or a {\bf cosmological constant}. There is also no real correlation between dark matter and dark energy, other than that we have determined their existance through their being {\it necessary}.

\subsection{Evidence}
We know that distance from the center of a circle and an object's orbital relation are related. It turns out if the center of the circle is the center of mass, the orbital speed {\it decreases} as we move away from the center. If the mass is distributed evenly, the orbital speed {\it increases}. Since the orbital speed of the stars within our galaxy increase as we move away from the galactic center, there must be a large amount of mass on the galaxy's halo. Since we can detect no radiation from it, we call that dark matter.

For galaxies that are not our own, we can make a similar calulation: using the mass-to-luminosty ratio, we can determine the total mass of the galaxy. Then, we can measure the velocities of stars and dust clouds in that galaxy and use the laws of galaxy to caluclate their mass. The difference in mass is dark matter.

We find that the composition of a spiral galaxy is typically 98\% or more dark matter.

We can apply these same techniques to galactic clusters. If we assume they orbit each other, the gravitational calculations predict a far greater mass than their luminosities would. Thus we see that Galactic clusters are even more than 90\% percent dark matter!

We can also measure the temperature of the hot gas (interstellar medium) within a galaxy by measuring the X-rays that medium emits. Since temperature is related to mass in this case, we can determine a galaxy's total mass with some calculations. Studies performed using this method see galaxies as containing more mass than luminosity would predict and thus agree with the above gravitational calculations.

We can also use {\bf gravitational lensing} to make the mass measurements. This technique relies on large masses ``bending'' light as it travels by exerting gravitational influences on the photons. By measuring the perceived shift, we can determine the mass of the objects between us and a source of light. By using this technique, we can use Einstein's Laws instead of Newton's. Since these results agree, we can increase our confidence in dark matter.

We are pretty sure that there are two options:
\begin{itemize}
\item our understanding of gravity is correct and dark matter exists, or
\item our understanding of gravity is incorrect.
\end{itemize}

That said, we are quite confident in our understanding of gravity. Furthermore, no one has been able to come up with an explanation which neatly explains our observations.

\subsection{Composition}
Dark matter may either be composed of particles we have already detected -- but in some form as to be undetectable -- or of exotic particles. At least most of it is likely exotic.

Dark matter could contain some non-exotic matter: if your body were in space, it would be undetectable as it would not be luminous enough to be visible. Similarly, planets, brown dwarfs, faint red M-sequence stars, etc are also classified as dark matter since they are too dim to be seen. That siad, if dark matter contained any of these objects, we could detect it: due to gravitational lensing, any of these objects passing in front of any source of light would be noticeable. The duration of this lensing would reveal the object's mass. We have discovered a few of these events, but not nearly enough to explain dark matter's prevalance or mass. Similar measurements agree dark matter can not be mostly comprised of black holes.

Models of nuclear fusion give us an estimate of the total number of protons, neutrons, etc in the galaxy. Their mass would comprise about one-sixth of the measured mass of the universe; thus there must be some exotic particles filling the five-sixths of the universe's mass.

We imagine dark matter to be composed heavily of {\bf weakly interacting massive particles}, or {\bf WIMPs}. These particles would be similar to neutrinos, in that they interact only with a couple of the four forces (ie. weakly interact) but far more massive and slower moving. Note that though these are refered to ass ``massive particles'', they are really subparticles and are thus only massive relatively speaking.

A large amount of WIMPs being present in the outer halo of a galaxy fits within our current understanding.

We have not yet detected any WIMPs, but through large-scale space particle detection and particle colliders, we are hopeful we will detect some soon.

\subsection{Dark Matter's Role}
Dark matter likely played an essential role in formating galaxies: by being os large in mass, areas high in dark matter likely attracted many other particles and eventually developed the mass to become a galaxy.

We also know that the universe is arranged into galactic clusters, superclusters, and even large sheets of superclusters. The reason mass in our universe is so highly divided is likely due to the effects of gravity from large amounts of dark matter. In fact, the current galactic structure likely mirrors the initial distribution of dark matter.

\subsection{The Fate of the Universe}
We can determine a {\bf critical density} of our universe by which a universe with a larger density will eventually start contracting and one with a smaller density will simply expand forever. Including dark matter, our estimates of the universe's matter content fall short of this critical density (we measure mass approximately equal to 0.5\% of the required amount and believe dark matter is 50 times more massive, thus we have 25\% of the required mass). Thus the universe seems likel to continue expanding forever, as we are doubtful there is more dark matter than we have predicted.

In fact, the expansion of the universe is {\it increasing} over time, which should not happen based on our understanding of gravity. Thus we label dark energy as the force causing the expansion.

\subsubsection{Expansion Patterns}
Given future changes in expansion rates, we determine four possible expansion patterns:
\begin{description}
\item[recolapsing] if there were no dark energy and the universe was above critical density, universal expansion would eventually reverse and end in a ``big crunch''. This is sometimes refered to as a {\bf closed universe}, since it could be modelled by a mathematically closed sphere in more dimensions.
\item[critical] if there were no dark energy and the universe was at critical density, the universe's expansion would slow over time but never reverse. Mathematicall, we could call this a {\bf flat universe}.
\item[coasting] if there were no dark energy and the universe was below critical density, the universe would keep expanding at its current rate forever. We could mathematically call this an {\bf open universe}.
\item[accelerating] if dark matter exerts a repulsive force which causes the universe's expansion to acclerate over time, the universal expansion rate would increase over time. This type of universe may be closed, open, or flat. Current evidence points to our universe being an accelerating flat universe.
\end{description}

Based on the average distance between galaxies over time, we seem to be in an accelerating universe. We measure this by looking at white dwarf supernovae: their distance tells us the lookback time and their redshift tells us what rate the galaxy had been expanding at.
