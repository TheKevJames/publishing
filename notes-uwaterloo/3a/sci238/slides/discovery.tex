\subsection{Discovering the Universe for Yourself}
\subsubsection{View from Earth}
With the naked eye we can see $>2000$ stars as well as the milky way. Some of these stars are sorted into 88 constellations. These lie on the celestial sphere (all stars lie on celestial sphere).
\begin{itemize}
\item the \textbf{ecliptic} is the sun's apparent path through celestial sphere
\item \textbf{north celestial pole} is directly above earth's north pole
\item \textbf{celestial equator} is projection of earth's equator on celestial sphere
\end{itemize}
The milky way is a band of light making a circle around the celestial sphere (just our view into the plane of our galaxy).

\subsubsection{Local Sky}
An object's altitude (above horizon) and direction (along horizon) specify its location in your local sky.
\begin{itemize}
\item \textbf{meridian}: line throuh zenith and connecting N and S points on horizon
\item \textbf{zenith}: point directly overhead
\item \textbf{horizon}: all points 90 degress from zenith
\end{itemize}

\subsubsection{Measurements}
\begin{center}
\begin{tabular}{|c|c|}
    \hline
    Angular size of sun & 0.5 degrees\\
    \hline
    Angular size of moon & 0.5 degrees\\
    \hline
    Width of finger &1 degree\\
    \hline
    Width of hand & 20 degrees \\
    \hline
    Width of fist & 10 degrees\\
    \hline
\end{tabular}
\end{center}
Arcminutes (denoted ') are 1/60 of a degree and arcseconds are 1/60 of arcminutes (denoted '').
\begin{center}
    angular size = physical size $\times \frac{360\,^{\circ}}{2\pi \times \text{distance}}$
\end{center}

\subsubsection{Star Rise}
Earth rotates west to east so starts appear to move east to west. Stars near the north pole are circumpolar and never set. Starts near south pole are not seen.
What constellations we see depends on latitude (but not longitude) because position on Earth determines which constellations remain below the horizon, and time of year because Earth's orbit changes the apparent location of the Sun among the stars.

The altitude of the celestial pole equals your latitude (ex if Polaris is $50\,^{\circ}$ above the horizon due north you are at latitude $50\,^{\circ}$N). All constellations move counter clockwise around Polaris.

\subsubsection{Seasons}
Direct sunlight heats more so in the summer we are angled toward the sun so we get more direct sunlight. Sun's position varies by season (summer has is higher). Earth's distance from Sun varies by at most 3\% so it cannot effect the temperature as much as the axis tilt can.
\begin{itemize}
\item Summer solstice: highest path (rise and set at most extreme north)
\item Winter solstice: lowest path (rise and set at most extreme south) for this half of the year the sun's angle at north pole is less than 0
\item Spring/Fall equinox: middle path (rise and set at exactly east and west) here the angle of sun at noon on north pole is earth's axis tilt.
\end{itemize}

Earth's axis rotates once every 26,000 years

\subsubsection{The Moon}
Moon's phases:
\begin{itemize}
\item new (6am to 6pm)
\item waxing crescent (glowey bit on the right) (9am to 9pm)
\item first quarter (noon to midnight)
\item waxing gibbous (3pm to 3am)
\item full (6pm to 6am)
\item waning gibbous (9pm to 9am)
\item last quarter (midnight to noon)
\item waning crescent (glowey bit on the left) (3am to 3pm)
\end{itemize}

A lunar eclipse occurs when earth casts a shadown across the moon. \textbf{Penumbra} is a glowing ring around (due to light refraction) \textbf{umbra} which is strict shadow.
\begin{itemize}
\item full lunar eclipse - moon passes through umbra
\item partial lunar eclipse - moon partially passes through umbra
\item penumbral lunar eclipse - moon passes through penumbra
\end{itemize}
Eclipses only occur wiht a full moon at night.

A solar eclipse only occurs at new moon during the day.

The Moon's orbit is tilted 5° to ecliptic plane. So we have about two eclipse seasons each year, with a lunar eclipse at new moon and solar eclipse at full moon.Eclipses recur with the 18-year, 11 1/3-day saros cycle, but type (e.g., partial, total) and location may vary.

\subsubsection{Ancient Planets}
\begin{itemize}
\item Mercury
\begin{itemize}
\item difficult to see; always close to Sun in sky
\end{itemize}
\item Venus
\begin{itemize}
\item very bright when visible, morning or evening ``star''
\end{itemize}
\item Mars
\begin{itemize}
\item noticeably red
\end{itemize}
\item Jupiter
\begin{itemize}
\item very bright
\end{itemize}
\item Saturn
\begin{itemize}
\item moderately bright
\end{itemize}
\end{itemize}
Ancients saw planets that moved eastward relative to the starts but occasionally went backwards (called \textbf{retrograde}). Impossible to explain with geocentric universe. The greeks rejected the heliocentric universe because they couldn't observe the stellar paralax (difference in position of a star as seen from earth) because the change in distance was too small to be measured.
