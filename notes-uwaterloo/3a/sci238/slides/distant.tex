\subsection{Other Planetary Systems: The New Science of Distant Worlds}
Its hard to learn about extrasolar planets because planets are too close to their star and too dark to easily see.

Planet Detection:
\begin{itemize}
\item Direct: pictures or spectra of the planets themselves
\item Indirect: measurements of stellar properties revealing the effects of orbiting planets
\end{itemize}

Stars wobble back and forth a bit because they are also pulled by their planets' gravity which we can use to reveal the mass (this is a lower limit since we would need to know its tilt to be accurate) and orbit of planets. These changes are very hard to see and we need to use doppler technique to measure them (accurate to 1m/s of movement).

The first extrasolar planet was discovered around 51 Pegasi when a 4 day orbital cycle was found. This means that the planet is very close to the star (it also has a mass similar to jupiter)

If you see a star with a wobble of greater than 1 year it has a planet closer than 1AU.

We can also watch for changes in brightness when a planet passes infront of the star called a transit. We use this to tell the planet's radius. The Kepler mission was launched to look for these tiny changes in brightness.

We can also use gravitational lensing to see how a planet's mass bends the light of a star and dust disks to see gaps in disks of dust and gas around stars where planets are.

We can also monitor the change in spectrum during a planet's transit across its star to know about its atmosphere composition. Similarly we can measure the surface tempertaure of the planet by seeing how the temperature of the star changes as the planet passes it.

Most detected planets have orbits smaller than Jupiter's but this is because planets at a farther distance are harder to detect. Some extrasolar planets have more eliptical orbits and tend to have greater mass than Jupiter (this is also because smaller mass planets are harder to detect).

Extrasolar planet suprprises:
\begin{itemize}
\item highly elliptical orbits
\item huge diversity of size and density
\item massive planets very close to their start (hot Jupiters)
\begin{itemize}
\item Nebular theory predicts that massive planets cannot form inside of frost line ($<<$ 5AU)
\item may be explained by planetary migration or gravitational encounters
\begin{itemize}
\item \textbf{planetary migration}: young planets spin very quickly and can form disks which can tug the planet's orbit inward
\item \textbf{gravitational encounters}: two massive planets getting two close to each other can result in one getting ejected into a highly elliptical orbit, multiple ones can cause a inward migration, this is due to one planet transfering energy and angular momentum to the other
\end{itemize}
\end{itemize}
\end{itemize}
The above weirdness caused a re-evaluation of Nebular Theory.
