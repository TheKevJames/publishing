\subsection{Science of Astronomy}
The seven days of the week were named after the sun, moon, and five visible planets (tues-mars, wed-mercury, thurs-jupiter, fri-venus, sat-saturn).

Some ancient time peices:
\begin{itemize}
\item Stonehenge
\item Templo Mayor (mexico)
\item Sun dagger marks solstice(US)
\item Macchu Picu
\item Polynesia were good celestial navigators
\item China first to record supernovas
\end{itemize}

\subsubsection{Ancient Greeks}
Greeks were the first people known to make models of nature. They tried to explain patterns in nature without resorting to myth or the supernatural.

Eratosthenes measured the size of earth in 240BC: Distance between two cities (Syene and Alexandria) was 5000 stadia, using a stick in a hole he could see that when the sun was directly overhead of alexandria it was off by 7 degrees in Syene. 7/360 * (circumference of earth) = 5000 stadia. Was only off by 2000km.

Greeks used geocentric model(heaves must be perfect with everything being perfect spheres and circles). This couldnt explain retrograde motion. Ptolemy came up with the most sphisticated geocentric model and it was accurate enough to be in use for 1500 years. He explained retrograde motion by having planets make little circles occasionally in their larger orbit of earth.

Which of the following is NOT a fundamental difference between the geocentric and Sun-centered models of the  solar system?
\begin{enumerate}
\item Earth is stationary in the geocentric model but moves around Sun in Sun-centered model.
\item Retrograde motion is real (planets really go backward) in geocentric model but only apparent (planets don't really turn around) in Sun-centered model.
\item Stellar parallax is expected in the Sun-centered model but not in the Earth-centered model.
\item \textbf{F} The geocentric model is useless for predicting planetary positions in the sky, while even the earliest Sun-centered models worked almost perfectly.
\end{enumerate}

Greek knowledge was preserved through:
\begin{itemize}
\item The Muslim world preserved and enhanced the knowledge they received from the Greeks.
\item Al-Mamun's House of Wisdom in Baghdad was a great center of learning around A.D. 800.
\item With the fall of Constantinople (Istanbul) in 1453, Eastern scholars headed west to Europe, carrying knowledge that helped ignite the European Renaissance.
\end{itemize}

\subsubsection{Copernican Revolution}
Copernicus:
\begin{itemize}
\item Proposed a Sun-centered model (published 1543)
\item Used model to determine layout of solar system (planetary distances in AU) But . . .
\item The model was no more accurate than the Ptolemaic model in predicting planetary positions, because it still used perfect circles.
\end{itemize}
Tycho Brahe:
\begin{itemize}
\item Compiled the most accurate (one arcminute) naked eye measurements ever made of planetary positions.
\item Still could not detect stellar parallax, and thus still thought Earth must be at center of solar system (but recognized that other planets go around Sun).
\item Hired Kepler, who used Tycho's observations to discover the truth about planetary motion.
\end{itemize}
Johannes Kepler:
\begin{itemize}
\item Kepler first tried to match Tycho's observations with circular orbits
\item But an 8-arcminute discrepancy led him eventually to ellipses.
\end{itemize}
Kepler's laws:
\begin{enumerate}
\item The orbit of each planet around the Sun is an ellipse with the Sun at one focus
\item As a planet moves around its orbit, it sweeps out equal areas in equal times  (means that a planet travels faster when it is nearer to the sun)
\item More distant planets orbit the Sun at slower average speeds, obeying the relationship: $p^2 = a^3$ where p = orbital period in years and a = avg. distance from Sun in AU
\end{enumerate}
Gelileo fixed some flaws in Compernican revolution:
\begin{itemize}
\item Earth could not be moving because objects in air would be left behind.
\begin{itemize}
\item Galileo's experiments showed that objects in air would stay with Earth as it moves, showed that objects will stay in motion unless a force acts to slow them down
\end{itemize}
\item Non-circular orbits are not ``perfect'' as heavens should be.
\begin{itemize}
\item Tycho's observations of comet and supernova already challenged this idea.
\item Galileo used telescope to spot imperfections (sunspots, and craters on the moon)
\end{itemize}
\item If Earth were really orbiting Sun, we'd detect stellar parallax.
\begin{itemize}
\item Galileo showed stars must be much farther than Tycho thought — in part by using his telescope to see the Milky Way is countless individual stars. If stars were much farther away, then lack of detectable parallax was no longer so troubling.
\end{itemize}
\end{itemize}
He also saw Jupiter's moons so not everything was orbiting earth. Proved Venus' orbit of the sun and used it to explain retrograde motion.

\subsubsection{Scientific Theory}
Science Hallarks:
\begin{itemize}
\item Modern science seeks explanations for observed phenomena that rely solely on natural causes.(A scientific model cannot include divine intervention)
\item Science progresses through the creation and  testing of models of nature that explain the  observations as simply as possible. (Simplicity = ``Occam's razor'')
\item A scientific model must make testable predictions about natural phenomena that would force us to revise or abandon the model if the predictions do not agree with observations
\end{itemize}

A scientific theory must:
\begin{itemize}
\item Explain a wide variety of observations with a few simple principles
\item Must be supported by a large, compelling body of evidence
\item Must NOT have failed any crucial test of its validity
\end{itemize}

\subsubsection{Astrology}
Astronomy is a science focused on learning about how stars, planets, and other celestial  objects work. Astrology is a search for hidden influences on  human lives based on the positions of planets  and stars in the sky.
