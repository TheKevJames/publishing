\subsection{Light and Matter: Reading Messages from the Cosmos}
How do light and matter interact?
\begin{itemize}
\item emission/absorption (determines brightness)
\item transmission
\item reflection (determines color)
\end{itemize}

\subsubsection{Waves}
Speed of light = wavelength x frequency\\
Photon energy = (6.626*10$10^-34$) plancks constant * frequency

Polarization is the direction a light wave is vibrating

\subsubsection{Matter}
Vocabulary:
\begin{itemize}
\item Atomic number = \# of protons in nucleus
\item Atomic mass number = \# of protons + neutrons
\item Isotope: same \# of protons but different \# of neutrons
\item Ionization: stripping of electrons, changing atoms into plasma
\item Dissociation: breaking of molecules into atoms
\item Evaporation: breaking of flexible chemical bonds, changing liquid into solid
\item Melting: breaking of rigid chemical bonds, changing solid into liquid
\end{itemize}

\subsubsection{Light}
Three types of spectra: emission line (specific elements emit light at certain wavelengths due to electrons jumping), continuous, absorption line (a cloud in between absorbes some light).

\subsubsection{Thermal Radiation}
Hotter objects emit more light at all frequencies per unit area. Hotter objects emit photons with a higher average energy.

Wein's law approximates starts to blackbody radiators (absorbs all kinds of radiation and emits energy regardless). Peak wavelength times the temperature is a constant (Plank radiation constant): $\lambda_{peak}T = 2.898\times10^3$. This also means that stars of shorter wavelength (blue) are hotter.

Interpretting Spectrum:
\begin{itemize}
\item look for parts of the visible spectrum that have lower intensity (low intensity blue light means it looks red and vice versa)
\item look for a spike in the infrared to indicate tempurature
\item look for absorbtion lines for the content of the atmosphere
\item look for emission lines to describe the upper atmosphere
\end{itemize}

\subsubsection{Doppler effect}
As something moves towards us its spectrum gets shifted
\begin{enumerate}
\item measure spectrum of stars composition in lab (heat gases)
\item measure spectrum of star
\item compare placement of lines
\item if wavelength of star is longer than lab (it shifted right/red) its moving away
\item $f_{star} = \sqrt{\frac{c-v}{c+v}} \times f_{lab}$
\end{enumerate}

When an object is rotating the width of its spectral lines can tell us how fast its spinning.
