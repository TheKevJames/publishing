\section{Chapter S4: Building Blocks of the Universe}

Quantum ideas
\begin{itemize}
    \item Light behaves light particles (photons)
    \item Protons and neutrons are not fundamental, they are made of quarks
    \item Antimatter can annihilate matter and produce pure energy) photons, neutrinos, etc
    \item Particles can behave like waves
\end{itemize}

Classes of particles
\begin{itemize}
    \item Bosons: photons, gluons
    \item Fermions
    \begin{itemize}
        \item Leptons: electrons, neutrinos
        \item Quarks: up, down, strange, charmed, top and bottom.  Proton = u,u,d; neutron = u,d,d
    \end{itemize}
\end{itemize}

Fermions have half-integer spin (spin is a multiple of $\frac{h}{2\pi}$).  Bosons have integer spin.

Each particle has an antimatter counterpart.  When a particle collides with its antimatter counterpart, they annihilate and become pure energy (photons) in accord with $E = mc^2$

\subsection{Fundamental Forces}

Four fundamental forces
\begin{itemize}
    \item Gravity: holds large-scale structures together. Exchange particle = gravitons
    \item Electromagnetic: holds electrons in atoms. Exchange particle = photons
    \item Strong: hold nuclei together.  Exchange particle = gluons
    \item Weak: mediates nuclear reactions. Exchange particle = weak bosons
\end{itemize}

Relative strength
\begin{itemize}
    \item Inside nucleus
    \begin{itemize}
        \item Strong force 100 times electromagnetic
        \item Weak force is $10^{-5}$ times electromagnetic
        \item Gravity is $10^-{43}$ times electromagnetic
    \end{itemize}
    \item Inside nucleus
    \begin{itemize}
        \item Strong and weak are unimportant
    \end{itemize}
\end{itemize}

\subsection{Uncertainty}

The uncertainty principle: The more we know about where a particle is located, the less we can know about its momentum, and the converse.

Electrons do not have discrete positions, we only know the probability of finding an electron at a certain spot.

Uncertainty in location * uncertainty in momentum = Planck's constant (h)\\
Uncertainty in energy * uncertainty in time = Planck's constant (h)

Exclusion principle: Two fermions of the same type cannot occupy the same quantum state at the same time.

Degeneracy pressure: Squeezing matter restricts locations of its particles, increasing their uncertainty in momentum.  Since two particles cannot be in the same quantum state (including momentum) at the same time, so there must be an effect that limits how much matter can be compressed.

``Electron degeneracy pressure'' is what supports white dwarfs against gravity.  ``Neutron degeneracy pressure'' is what supports neutrons stars against gravity.

Quantum tunneling is when the effect of the uncertainty of subatomic particles allows them to ``tunnel'' through barriers.  This is how protons can fuse, they tunnel through the electromagnetic energy barrier.


Virtual pairs are matter-antimatter pairs of particles that rapidly appear and disappear.  The combined energy of these pairs is called the \textbf{vacuum energy}.  Particles can be produced near black holes if one member of a virtual pair falls into the black hole.  Energy to permanently create the other particle comes out of the black hole's mass --- the black hole slowly evaporates.
