\subsection{Making Sense of the Universe: Understanding Motion, Energy, and Gravity}
Momentum = mass X velocity.

Angular momentum = mass x velocity x radius

Remember that force is based on acceleration not speed.

Remember that weight force and mass is matter

Newtons Laws:
\begin{itemize}
\item An object moves at constant velocity unless a net force acts to change its speed or direction.
\item Force = mass x acceleration
\item For every force, there is always an equal andopposite reaction force
\end{itemize}

Conservation Laws:
\begin{itemize}
\item conservation if momentum
\begin{itemize}
\item The total momentum of interacting objects cannot change unless an external force is acting on them.
\item Interacting objects exchange momentum through equal and opposite forces.
\end{itemize}
\item conservation of angular momentum
\begin{itemize}
\item The angular momentum of an object cannot change unless an external twisting force (torque) is acting on it.
\item Earth experiences no twisting force as it orbits the Sun, so its rotation and orbit will continue indefinitely.
\end{itemize}
\end{itemize}

Types of energy:
\begin{itemize}
\item Kinetic (motion)
\item Radiative (light)
\item Potential (stored)
\item Thermal: The collective kinetic energy of many particles
\begin{itemize}
\item Temperature is the average kinetic energy of the many particles in a substance
\end{itemize}
\item Gravitational Potential
\begin{itemize}
\item In space, an object or gas cloud has more gravitational energy when it is spread out than when it contracts.– A contracting cloud converts gravitational potential energy to thermal energy
\item $F_g = G\frac{M_1 M_2}{d^2}$
\item items orbit around their center of mass (usually close to the big thing they are orbiting)
\end{itemize}
\item Mass: $E = mc^2$
\begin{itemize}
\item A small amount of mass can release a great deal of energy (for example, an H-bomb).
\item Concentrated energy can spontaneously turn into particles (for example, in particle accelerators).
\end{itemize}
\end{itemize}

We combine Newton's law of gravity and Keplers orbital law:\\
\begin{center}
$p^2 = \frac{4\pi^2}{G(M_1 + M_2)}a^3$\\
p = orbital period\\
a = average orbital distance\\
M = object masses
\end{center}

The total orbital energy is the sum of gravitational and kinetic and remains the same (which is why we go faster closer). This means its very hard to change an orbit.

Earth's escape velocity is 11km/s (40000km/h)

Tides:
\begin{itemize}
\item caused by moon being closer to one side
\item depends on phase because of sun's position
\item tidal friction slows earths rotation, making moon get farther from Earth
\end{itemize}
