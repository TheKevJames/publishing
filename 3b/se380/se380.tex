\documentclass[12pt]{article}
\usepackage{amsmath,amssymb,bookmark,parskip,custom}
\usepackage[margin=.8in]{geometry}
\allowdisplaybreaks
\hypersetup{colorlinks,
    citecolor=black,
    filecolor=black,
    linkcolor=black,
    urlcolor=black
}
\setcounter{secnumdepth}{5}

\begin{document}

\title{SE 380 --- Introduction to Feedback Control}
\author{Kevin James}
\date{\vspace{-2ex}Fall 2015}
\maketitle\HRule

\tableofcontents
\newpage

\section{Introduction}
There are two main types of control:
\begin{description}
\item[Open-Loop control] simply converts reference inputs to control inputs (through a controller), then combines these with disturbance inputs in the plant to provide outputs.
\item[Feedback control] adds a second susbsytem, where outputs are fed back into a sensor which measures these outputs, compares them to the reference input, and generates an error signal (which we attempt to minimize)
\end{description}

\subsection{Potential Advantages of Feedback Control}
\begin{description}
\item[Tracking] can make output follow reference input.
\item[Regulation] compensates for disturbance inputs.
\item[Robustness] compensates for variation in plant dynamics.
\item[Stabilization] can potentiallially stabilize unsafe plants.
\end{description}

\end{document}
