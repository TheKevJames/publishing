\section{Chapter 12 -- Star Life Cycle}
All elements heavier than hydrogen and helium were created through fusion or supernova.

\subsection{Star Birth}
Gravity causes a gas cloud to contract until it's hot enough to sustain fusion.

Heat generated from converting gravitational potential.

The cloud's internal gas pressure resists gravity.

{\bf gravitational equilibrium}: gravity force pulling inward matches radiative pressure pushing out, star size is stable

Two factors:
\begin{itemize}
\item higher density, more material per space (still low enough to be strong vacuum on earth)
\item lower temperature, reduces pressure (typically 10-30 K)
\end{itemize}

\[ M_{min} = 18 * M_{Sun} * \sqrt{T^3 / n} \] where T = temp of gas in K and n = density of gas in particles per cubic meter

These clouds are referred to as molecular clouds since they are cold enough for H atoms to pair up.

They are very large, many stars born in simulatneous clusters.

\subsubsection{Protostar Stage}
As it compresses, the gas starts to heat up. The cloud takes a lumpy, clumpy shape.

Energy radiates away until it's dense enough to trap heat. Temp rises hard.

At this point we are still not hot enough for fusion.

The cloud rotates rapidly from conservation of momentum of particles.

Flattens to a disk from particle collisions (planets may form here eventually)

May shoot jets perpindicular to disk (not sure why, we suspect magnetic fields due to rotation)

Field also generates protostellar wind (stronger version of solar wind)

Wind and jets shed momentum by expelling material, slowing rotation

Angular momentum also cause of binary star systems
\begin{itemize}
\item stars form close together
\item orbit instead of crash
\item more momentum = larger orbit)
\end{itemize}

The protostar becomes a true star at {\bf 10 million K}, continues to rise until balance achieved (energy from fusion = energy radiated)

Time to reach main sequence phase proportional to mass, large stars are faster

Stars range in size, over 99\% are within 0.5 and 2 $M_{Sun}$ (leaning below 1)

Large ones burn out faster

Stars:
\begin{itemize}
\item can't be more than $300 M_{Sun}$ because it would blow off its outer layers
\item can't be less than $0.08 M_{Sun}$ or it wont get hot enough, stabilizes as a brown dwarf
\item brown dwarf gravity collapse halted due to degeneracy pressure, restriction on how close elections can be together
\end{itemize}

\subsection{Low Mass Stars ($<8 M_{Sun}$)}
Main-Sequence stage is 90\% of star's lifetime

Star regulates itself: if fusion works too fast, core expands until it cools again

\subsubsection{Red Giant stage}
When core hydrogen is depleted, fusion will cease

No more radiation pushing outwards, core shrinks from gravity

Core is inert helium, small shell of hydrogen around it fuses (higher rate than core), outer layers expand

Star is 100 times larger and 1000 times brighter than main sequence stage

Weaker gravity at surface, increased stellar wind

Fusion shell makes more helium, core gets heavier and shrinks more, shell gets even hotter and denser

\subsubsection{Helium Core Fusion stage}
Feedback loop until core reaches 100 million K, helium start fusing into carbon

At this point thermal pressure is too low (gravity is fucking intense)

Core sustained by degeneracy pressure, which does not increase with temperature

Helium fusion heats the core without causing it to inflate, fusion rate spikes (called helium flash)

So much so that thermal pressure becomes dominant and core increases in size, lowering temp and fusion

Outer layers shrink again, stabilizes back at yellow

This stage is short, 1\% of star lifetime

\subsubsection{Last Gasps}
When helium runs out carbon core shrinks again

Outer layer expansion again from helium shell fusion (hydrogen shell still going, core double layered)

Now even larger than red giant stage

Star is too low mass to fuse carbon, will not reach 600 million K

Too large for its mass, gravity too low on surface, outer layers start being blown off

Forms planetary nebula (nothing to do with planets), bright glowing ring

Will combine into interstellar dust when cooled

Exposed core remains as a stable white dwarf, gas recycled into a new star

Will cool until it no longer emits light, then sit in the dark of space

\subsection{High Mass Stars}
\subsubsection{Hydrogen Fusion}
Once in main stage, protons can slam into carbon, nitrogen, oxygen molecules with enough energy

Follows {\bf CNO cycle} of fusion and decay
\begin{enumerate}
\item $C^{12} + H = N^{13}$
\item $N^{13} \to C^{13}$
\item $C^{13} + H = N^{14}$
\item $N^{14} + H = O^{15}$
\item $O^{15} \to N^{15}$
\item $N^{15} + H = C^{12} + He^{4}$
\end{enumerate}

This cycle allows hydrogen fusion to proceed much faster than proceed much faster than typical proton chain (more valid things to bump into)

Makes these stars much brighter, lives much shorter

\subsubsection{Becoming a Supergiant}
Reaches hydrogen fusing shell stage much faster, outerlayers expand

Temperatures so high that degeneracy pressure never takes over, no helium flash (gradual, like starting hydrogen fusion was)

Fuses helium into inert carbon core in just a few thousand years

Core fusion stops, core shrinks, helium shell forms, surface expands

Alternates between shrinking and expanding as core reaches next level of fusion

Helium $>$ Carbon $>$ Oxygen $>$ Neon $>$ Magnesium $>$ Silicon $>$ Iron

The biggest stars transitions so quickly other layers dont have time to resond, become red supergiant

{\bf eg.} Betelgeuese, Orion's left shoulder. 500 solar radii, 2 AU

\subsubsection{Heavier Nuclei}
Simplest heavy fusion is helium-capture reaction
\begin{enumerate}
\item $C^{12}+ He^{4} = O^{16}$
\item $O^{16}+ He^{4} = Ne^{20}$
\item $Ne^{20}+ He^{4} = Mg^{24}$
\end{enumerate}

Note that each transition upwards drains the core and causes another shell to form

All shells will be active simultaneously

Once hot enough, can start fusing those heavy nuclei
\begin{enumerate}
\item $C^{12} + O^{16} = Si^{28}$
\item $O^{16} + O^{16} = Si^{31} + H$
\item $Si^{28}+ Si^{28} = Fe^{56}$
\end{enumerate}
\subsubsection{Iron, the Dead End}
Iron is the only element where it is not possible to generate nuclear energy, fusion or fission

Lowest mass per nuclear particle of all elements

Iron core can only resist gravity through degeneracy pressure, but more iron keeps piling on

Then gravity pushes past the quantum mechanical limit

\subsubsection{Supernova}
Electrons disappear by combining with protons to form neutrons, realeasing neutrinos

Iron core with a mass near $M_{Sun}$ and radius larger than Earth collapses into a ball of neutrons just a few kilometers across in a fraction of a second

Stops due to neuton denegeneracy pressure

This neutron star is similar to an atom nuecleus the size of Kitchener

The gravitational collapse of the core releases an enormous amount of energy, more than 100 times than the Sun will radiate over its entire 10 billion year lifetime

Old theory was that supernova was caused by matter collapsing into neutron star and bouncing

New theory is collapse causes so many neutrinos to be formed that, despite how rarely they interact with matter, entire star is blown away

So hot that it's as bright as moderately sized galaxies for a few weeks, continue to expand and cool

Will eventually be incorporated into new stars in other gas clouds

Crab nebula is remnant of supernova from 1054 AD

if gravity is still strong enough to overcome neutron degeneracy pressure, collapses continues further to a black hole

Interesting note: due to larger stars dying and adding heavier elementss to the interstellar dust, newer stars have higher percentages of heavy elements than older stars (2-3\% vs 0.1\%)

Interesting note: most heavy elements are made in helium capture which adds two protons, so even numbered elements are more abundant in the universe

\subsection{Binary Systems}
The two stars exert tidal forces on eachother, create football shapes. When the more massive star begins to expand, the gas on the surface experiences strong pull to the other star than its own core, begins a mass exchange. May transfer back when ``soon to be as or more massive'' star begins expanding too.
