\subsection{Scale of the Universe}
\subsubsection{Objects in the Universe}
\begin{itemize}
\item Star:  A large, glowing ball of gas that generates heat and light through nuclear fusion
\item Planet: A moderately large object that orbits a star; it shines by reflected light. Planets may be rocky, icy, or gaseous in composition.
\item Moon: An object that orbits a planet
\item Asteroid: A relatively small and rocky object that orbits a star
\item Comet: A relatively small and icy object that orbits a star
\item Solar System: A star and all the material that orbits it, including its planets and moons
\item Nebular: An interstellar cloud of gas and/or dust
\item Galaxy: A great island of stars in space, all held together by gravity and orbiting a common center
\item Universe: The sum total of all matter and energy; that is, everything within and between all galaxies
\end{itemize}

\subsubsection{Light Travels}
Light travels at a finite speed (3000,000 km/s) so the farther away we look the farther back in time we look. Light years are distance light travels in a year (9,460,000,000,000 km).
\begin{center}
\begin{tabular}{|c|c|}
    \hline
    \textbf{Destination} & \textbf{Light travel time}\\
    \hline
    Moon & 1s\\
    \hline
    Sun & 8s\\
    \hline
    Sirius & 8 years\\
    Andromeda & 2.5 million light years\\
    \hline
\end{tabular}
\end{center}

\subsubsection{The Universe is Big}
We can't see a galaxy 15 billion light years away (universe is 14 billion years old) because looking 15 billion light-years away means looking to a time before the universe existed.

If we reduce the size of the solar system by a factor of 10 billion the sun is the size of a grapefruit and earth is the size of a ball point (15m from the sun)and alpha centauri is 2500 miles away

The milky way galaxy has about 100 billion stars. There are around 100 billion galaxies in universe. There are more stars in the universe than grains of sand on earth. It would take more than 3000 years to count the stars in the Milky Way Galaxy at a rate of one per second, and they are spread across 100,000 light-years.

The matter in our bodies came from the Big  Bang, which produced hydrogen and helium. All other elements were constructed from H  and He in stars and then recycled into new  star systems, including our solar system. On a cosmic calendar that compresses the history of the universe into 1 year, human civilization is just a few seconds old, and a human lifetime is a fraction of a second.

\subsubsection{Earth Moves Through Space}
Earth orbits the sun at an average distance of 1AU = 150 million kilometers (at 107,000 km/h) and tilted by $23.5\,^{\circ}$ rotating clockwise.

The sun moves 70,000 km/h and orbits the galaxy every 230 million years.

Galaxies in our Local Group are moving away from us and the farther a galaxy is the faster it is moving, which implies that we live in an expanding universe.
