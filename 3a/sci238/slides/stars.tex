\subsection{Searching the Stars}
\subsubsection{Brightness}
The brightness of a star depends on both distance and luminosity (amount of power a star radiates in energy per second, ie. Watts). The apparent brightness is the amount of starlight that reaches Earth in energy per second per square meter.

In concentric spheres, the amount of luminsoty through each sphere is identical, ie. we can divide luminosity by area to find brightness ($b = \frac{l}{2\pi d^2}$).

The brightness at a distance three times farther, then, is, one-ninth as much.

Given the parallax angle $p$ in arcseconds, the distance in parsecs is the inverse of $p$ and the distance in light-years is $d = 3.26 * \frac{1}{p}$.

We define magnitude $m$ and apparent magnitude $M$ as and ratio in luminosity is equal to $(100^\frac{1}{5})^{M_1 - M_2}$ and the ratio in apparent brightness is $(100^\frac{1}{5})^{m_1 - m_2}$. Thus, the birghter a star is, the lower its magnitude. We can think of magnitude as a ``ranking'' of stars by brightness.

\subsubsection{Star Spectrums}
Absorption lines in a star's spectrum tell us its ionization level. These lines also correspond to a spectral type which revels its temperature (from hottest to coolest: O B A F G K M).

\subsubsection{Thermal Radiation}
Hotter objects emit more thermal radiation at all frequencies.

The hottests stars are approximately $50,000K$, the coolest are $3,000K$. Ours is roughly $5,800K$. Note that these are surface temperatures: our Sun's core has a temperature of roughly ten million Kelvin.

Note that the mass of these suns range from 0.08 to 100 timex the mass of our Sun.

The life expectancy of our star is 10 billion years. A star ten times more massive uses $10^4$ times as much fuel, so lasts only 10 million years.

\subsubsection{Binary Systems}
About half of all stars are in binary systems.

\subsection{Patterns Among Stars}
An H-R diagram plots the luminosity and temperature of stars. Most stars fall on its main sequence.

Detailed modeling of the oldest global clusters reveal they are about 13 billion years old.

\subsection{Stellar Nurseries}
Stars form in dark clouds of dusty gas in interstellar space. The gas between stars is called the {\bf interstellar medium}.

The molecular clouds -- which contain the bulk of matter in interstellar space -- have a temperature of ten to thirty Kelvin and densities of approx 300 molecules per cubic cm.

Long wavelength light such as infrared light passes through these clouds more easily than visible light; this is why we can see the center of the Milky Way only with infrared light.

Gravity can create stars only if it can overcome the force of thermal pressure within the cloud. A typical cloud must contain at least a few hundred solar masses to overcome this pressure.

Gravity within a contracting gas cloud becomes smaller as the gas becomes denser; thus it can cause the cloud to break apart into fragments which may each form a star.

As contraction packs molecules closer, it becomes difficult for infrared and radio photons to escape. Thermal energy and pressure then build up. This slows down contractions, and the center of the cloud fragment becomes a {\bf protostar}.

Protostars \emph{must} be rotating in order to form planets.

More low-mass stars tend to form than high-mass ones.
