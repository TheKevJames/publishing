\subsection{Space and Time}
Einstein's Theories of Relativity:
\begin{itemize}
\item Special Relativity: usual ideas of space and time change as we approach the speed of light ($E = mc^2$)
\begin{itemize}
\item no object can travel faster than light
\item observing a object near the speed of light:
\begin{itemize}
\item time slows down
\item length contracts in direction of motion
\item mass increases
\end{itemize}
\item simultaneousness changes based on your frame of reference
\end{itemize}
\item General Relativity: new views of gravity
\end{itemize}

Motion is relative. Usually how fast your percieve something is based on your velocity compared to it. The exception is light which always is seen at the same speed (called \textbf{absolute relativity})

Postulates of special relativity:
\begin{itemize}
\item laws of nature are the same for everyone
\item speed of light is the same for everyone
\end{itemize}

Time Dilation: \[ t_1 = t_0\sqrt{1-\bigg(\frac{v^2}{c^2}\bigg)} \]
Length Contraction: \[ l_1 = l_0\sqrt{1-\bigg(\frac{v^2}{c^2}\bigg)} \]
Mass Increase: \[ m_1 = \frac{m_0}{\sqrt{1-\bigg(\frac{v^2}{c^2}\bigg)}} \]

Since no information can be transfered faster than the speed of light objects traveling near the speed of light will perceived information at different rates since the information is moving much more slowly relative to their speed.

\subsubsection{Tests for Relativity}
Michelson-Morley experiment found evidence for the absoluteness of the speed of light in 18887.

Time dilation occurs often to subatomic particles in accelerators.

Time dilation discovered with airplanes and very precise clocks.

$E = mc^2$ verified by measurements taken of the sun.

If the speed of light were not absolute light coming from a car moving towards you would travel at 100km/hr + c and a car moving parallel to you would be see at 100 km/hr so witnessing their collision would look very odd.
