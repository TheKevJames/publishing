% The Seventh Day of the Ninth Moon, Year 735

\section*{The Damned Murder House}
We awoke somewhere completely different. The country of Barovia, it turns out, which I don't recall ever hearing about. None of the others claim to have any idea how we got here, or where here is, and I'm inclined to believe them: the more time we spend in Barovia, the more I realize no one would ever come here of their own free will.

We were in a forest, like when we'd fallen asleep, but all the trees were\dots\ different, somehow. Ominous. Quiet. None of them spoke, or at least they didn't respond when I asked them where we were. A frusterated \textsc{Fire Bolt} thrown at them got swallowed by the fog before even reaching a tree.

Did I mention the fog? Or, as I consider it in my head, the Fog. Capitalized. Something about the Fog urged us forwards, urged us toward a small town that was utterly abandoned, as far as we could tell. It turns out this town was also named Barovia, though it didn't seem nearly big nor important enough to be named after the entire country. Or vice-versa, we never did ask about that.

The Fog urged us toward a decrepit house. Everything here was run-down, but this was the worst of them all: the entire house had a feeling of despair, of something evil we couldn't put our fingers on. Two children stood outside it, but something about them, as well, was\dots\ off.

The children convinced us to enter the house, to slay the ``monster downstairs'' and save their baby brother upstairs. Scathac, everyone's favourite hero, took up the quest to save them immediately. The rest of us were less convinced, but the Fog did a good job of ushering us inside. The inside of the house was surprising to me: nothing seemed out of place at first, everything was well-maintained, at least on the first few floors. Not that we spent much time looking around; with there being a baby trapped upstairs our nble hero charged up the staircase with nary a thought of proceeding with caution.

Ugh.

The rest of us followed, having to hurry just to keep her in our sights. Once we reached the third floor, that was when we found the first hints of something being out-of-place.

Everything was -- very suddenly -- dirty, decrepit, and broken down, like the rest of the town. And everything was \emph{angry}.

First, a set of armor animated itself and attacked us just as we'd reached the third floor landing. That was a brutal fight, Emmeral had a sword torn out of his hand and Scathac nearly broke her hand punching the suit of iron. We destroyed it enough to prevent it from hitting us eventually, but it certainly took a huge amount of effort on all our parts.

Then, what a nightmare, we came across a ghost, a vengeful spirit that attacked us as we entered the baby's nursery. It didn't seem to want to leave the room we found it in, which could have been an easy way to avoid fighting it, but no, Emmeral just had to kill the thing. Again. Gah, I was practically fighting against him the whole time: every time he opened the door to attack the thing, I immediately slammed the door shut with \textsc{Mage Hand}. If we'd have coordinated, it might have actually been a good strategy: close the door before the ghost has the time to counter-attack him. As it was, we just got in each other's way and pissed each other off.

Eventually, he actually managed to kill (re-kill?) it. At this point we confirmed what we already suspected: there was no baby in the nursery, the children had lied to us. In the mean time, the other three (did I mention Kinnaeia yet? the Dwarf cleric we'd saved from the Goblins, the most stereotypically good person in our party, who keeps stopping the group to check if she recognizes any of the religous symbology around here -- spoiler: she hasn't yet) somehow ended up fighting a broom.

Yes, a broom. And, honestly, I'm pretty sure Kinnaeia was losing.

Emmeral and I entered the hallway to see this and just stopped, unable to believe our eyes. It was\dots\ one of the more hilarious things I'd ever seen. We ended up rolling our eyes and going to explore the master bedroom; I didn't find anything and Emmeral only found an empty jewelery box. The box itself was worth some gold, at least, so it wasn't completely worthless.

Eventually, we convinced the others to leave the broom behind and continue exploring the house. We started noticing a bunch more oddities; skulls subtly stitched into curtains, oddly decaying papers, and things like that. This house was getting creepier by the minute.

Emmeral and I ended up exploring a study while the others explored a music room. This ended up being\dots\ pretty fortunate for me. I found a secret passageway in the back of a bookshelf which led to a half-opened chest with an adventurer laying dead over top of it, a letter clutched in one hand.

After carefully checking the area for further traps, I read the letter:

\begin{quote}
My most pathetic servant,

I am not a messiah sent to you by the Dark Powers of this land. I have not come to lead you on a path to immortality. However many souls you have bled on your hidden altar, however many visitors you have tortured in your dungeon, know that you are not the ones who brought me to this beautiful land. You are but worms writhing in my earth.

You say that you are cursed, your fortunes spent. You abandoned love for madness, took solace in the bosom of another woman, and sired a stillborn son. Cursed by darkness? Of that I have no doubt. Save you from your wretchedness? I think not. I much prefer you as you are.

Your dread lord and master,
Strahd van Zarovich
\end{quote}

That answered a lot: the house contained some hidden sacrificial altar, the ghost that attacked us was probably the woman this ``pathetic servant'' cheated with, and the son that we went to save was never even born to begin with. That said, it also raised some questions. Strahd van Zarovich? Who was Strahd?

Now isn't that an interesting question. I learned a lot more about Strahd eventually, but I'll continue in order for now.

Within the chest was three scrolls containing one-time use spells, deeds to the house and a windmill, a will stating those deeds should pass to the children we met outside, and the blank books which I am currently writing in. All of these aged immensely upon leaving the house, but they're still mostly intact.

I decided to keep the deeds a secret from the others. The fact that this Durst family had some magical scrolls, books about rituals everywhere, and something weird and magical going on\dots\ there may be some interesting things to find in that windmill, and now it's mine. At some point, I'll need to go investigate.

I emerged from the secret room to find Emmeral rummaging through the desk in the study proper. He found an iron key, but nothing else of value. I urged him out of the room to share the secrets I'd found once we were within earshot of the others. When his back was turned, I grabbed the wax stamp of the Durst family in case I needed to forge some papers to prove my ownership of the windmill.

Once we'd met the others in the second floor hallway, I shared the letter with them. I also told them about the scrolls (which they so graciously let me keep to myself, as our only mage) and the blank books. Mostly, they were interested in the letter. None of them seemed to make the connection to there being a hidden altar in this house (proved later when everyone was surprised), nor about the affair being the ghost we killed (though they realized that later). The baby was immediately remarked upon, but no one seemed to care who this Strahd person was. The general consensus was that Strahd was the god that this servant worshipped.

I find it interesting to note that none of them made the same connections I did. Even our cleric, who is self-professed to be an expert in religion, made no comment about Strahd not being the name of a god she'd ever heard of. I'll be keeping in mind the fact that she is less proficient here than she claims. Otherwise, their responses make sense: the monk and the druid don't seem particularly insightful and Emmeral always plays his cards close to his chest. By now, at least, I'm sure he's made the connections.

One thing that stuck out to me was how incredibly certain Kinnaeia was that both me and Emmeral were lying about something. I, of course, was pretending I had not found the deeds, but there was no way she should have been able to guess that. Even after she asked me directly and I lied in return, she seemed suspicious; but I'm confident I am a better liar than she is insightful into this. I need to watch this further, she seems to have some sort of supernatural insight into when Emmeral or I lie to her. I've seen other people lie to her, though, and she hasn't picked up on it. Something about this isn't right.

After this, we went downstairs to investigate the other floor we'd rushed past in our hurry to save the never-born child. Not much of interest here: a kitchen, which let us stock up on rations (which spoiled as we left the house), and a den containing some weapons. I got myself a fancy crossbow and some ammunition. I'm not sure when I'd ever need that over my spells, but it doesn't hurt to be prepared.

This is about the time we decided to leave the house. We'd been attacked three times, now, we'd seen the letter warning us of even more danger to come, and the children had lied to us as to there being a heroic reason for us to be there anyway. We went to leave through the main door we'd entered through, and found it missing.

Yes, missing. The door was now a wall.

We got a bit frantic after that, I'll admit. We raced through the house, re-examining every room we'd been through. We tore aside curtains over windows, tried to open doors to the upper floor balconies, but everywhere we had previously been able to see outside was replaced with a plain brick wall. In our frenzy, I'm confident we left no room unturned. It was only after we'd examined every possible exit and come up empty that we reconvened to discuss our next move.

We decided to continue searching the house, looking for the entrance to the basement. We hadn't found it yet, so we knew it was hidden, but we didn't yet know where.

Again, we examined every single room. Eventually, we found a secret passage behind a mirror in the nursery. It led upward, though, which certainly wasn't what we were looking for. Nonetheless, we trudged up the stairs, my \textsc{Dancing Lights} leading the way. There was a locked door at the top of these stairs, but the key Emmeral had grabbed from the study unlocked it. Note to self: next time we do something like this (if, Labelas forbid, we do this again), make sure to continue looting the whole area as we go.

The attic itself contained several rooms, but the most interesting was the children's room. Two skeletons, of roughly the same size, shape, and age as the children we'd met outside (hint, hint) held each other in the centre of the room. This is about when Scathac and Kinnaeia realized the children weren't real. Past them sat a dollhouse which seemed to be a perfect replica of the house. I moved to investigate this, but was taken aback by ghosts of the children appearing as I touched the playhouse.

The ghosts didn't seem to realize they were ghosts. They acted as they would have in life as young children; they were scared and afraid and didn't want to be left alone again. I convinced them to tell us how to get to the basement (secret staircase in a nearby room) and that we'd go kill the monster in the basement and then come back for them (yeah, right), but they got angry as we attempted to leave. As we left the room, the ghosts got so angry they launched themselves at Emmeral and I.

Let me tell you, it's not fun being possessed. The girl ghost possessed me and I could feel her influence upon me for far too long. She was bossy, far more than I was, and that pushed me to insist I lead the group, I walk through doors first, and everyone obey me. It annoys me that none of them seemed to notice how out of character this was for me, since I'm hardly the first person to put myself into danger.

By comparison, Emmeral, who'd been at the front of the group for the most part, had a similar experience being possessed by the boy ghost; he became a mopey, sad shadow of himself that I had to bully into doing anything. Which I \emph{enjoyed}, since the sister clearly enjoyed leading her brother around.

Nightmare.

I led the group into the room with the secret passage and forced Emmeral to look around; the room was filled with furniture covered by old cloths which I had the others uncover and investigate. A chest sat along one wall, which Scathac discovered contained a bundle of blood and bones -- that of the nursemaid. We shrugged and left down the staircase to the basement.

It didn't occur to me at the time, but writing this now\dots\ we were certainly callous to the idea of there being a woman's blood and bones in our very hands. This whole adventuring business is certainly making me less squeamish -- this isn't necessarily a bad thing, but that certainly didn't take long.

\subsection*{The Damned Murder House's Damned Murder Basement}
The moment we entered the basement, chanting filled the air. We couldn't make out the words, or even where it was coming from, but it continued throughout our entire time exploring the place. It was certainly a huge relief when the chanting stopped (much later).

Around the entrance to the basement was a set of crypts. Four of them (one for each family member) were sealed, and a fifth for the dead baby was unsealed and unoccupied. The sixth was more terrifying: it appeared as if someone had broken out of it, shoving the heavy stone slab out of the way as it left. We kept a close eye out for undead after this.

Past the crypts was a dining hall covered in gnawed upon bones. The family, then, was clearly cannibalistic, eating those poor folks that they sacrificed on their hidden altar. Elwing rushed into this room, impatient at our slow progress, and nearly died for it; a monster hid in a darkened alcove and bit his face almost immediately. Together, the rest of us managed to take it down, my magic this time manifesting as a \textsc{Witch Bolt}.

The others helped Elwing get back onto his feet while I pondered the new spell; shooting lightning out of my fingertips was awesome, and it certainly fried the creature to a crisp. That said, I'd meant to shoot \textsc{Magic Missile}s at the beast\dots\ I'm still not sure why, but the new spells I've learned since adventuring around have all developed as I attempted to cast something similar, but with the new spell having a greater effect. This is worth investigating; it's as if my magic is responding to my need for certain spells to be added to my repertoire. This is something I'll consider tonight, it would certainly be nice to have some way of ceasing the incessant howling keeping Morriana awake and annoying the rest of us. Maybe something to hold them back? Or to ward an area, keeping everything away? Hmm\dots\ I'll think more on this and explain later.

After reviving Elwing and telling him off for running ahead, we continued along -- a bit more cautiously, this time. We eventually came across a room containing a tall statue holding a crystaline orb. This\dots\ may have excited me more than it should. In retrospect my actions were idiotic, but at the time\dots\ the orb looked magical to me and thoughts of safety left my mind as I raced to examine it. Immediately upon touching the orb, though, ghosts appeared all around me and immediately attacked.

I was out in a second, unconscious from their attacks. When I was revived by the others, the ghosts were gone and a hole had appeared in one of the walls. Elwing had apparently cast a single spell that destroyed them all as well as the wall. At the time I wasn't sure how much of that was hyperbole and simply resolved to watch him further. Now, I know it was not: some of his abilities are incredibly potent, he's just not the best at using them at the right time.

Hmm, I should work on directing him more carefully. Maybe if I can convince him to listen to my orders in battle, we can use his abilities more effectively.

I was incredibly sad to notice the orb was missing. Scathac looked me straight in the face and explained that it had shattered during the fight. That lying \emph{nadorhuan} just wanted to keep it from me; it turned out that Elwing had taken the orb before reviving me. Between the three of them, Elwing, Scathac, and Kinnaeia had certainly been aligning themselves against me. Emmeral' current plan to abandon them is seeming like a better and better idea as I continue to recount this.

Anyway, at the time I had no idea about the orb and continued along a bit sadly, but otherwise as if nothing had happened. I was still possessed here and dictated our next steps of heading out a certain passage. This passage led to a bedroom, which contained \emph{yet another} set of surprises.

In the bedroom, we found some miscellaneous useless items like torches, a lantern, armor, and such. Our cleric was happy to see this, since she still had very little equipment after her run-in with Goblins, but the other items were much more interesting: a magical cloak which I was held back from touching given the\dots\ incident\dots\ in the previous room (which Elwing took -- the cloak seems to help protect him from enemies, I would love to get my hands on that), some potions of healing (I pocketed two and gave the other two to some more injured party members, who quaffed them immediately), some thieves' tools (Emmeral took these, the rest of us probably would have broken them anyway), and -- most importantly -- a Wizard's Spellbook.

Yes, the spellbook I mentioned earlier. The one I'm currently tearing my hair out attempting to decipher. The one that's been on my mind since I first laid eyes on it. That spellbook.

With some trepidation from our resident lying hero, I convinced the group to let me have the book. That was a huge victory -- worth losing the chance of getting the cloak over. What's one magic item compared to the chance of learning more about manipulating the very fabric of magic itself?

Unfortunately, picking up these items spawned a trap in the room; two ghasts burst out of the walls to attack us. As an aside, these were clearly the parents of the Durst family.

It was during this point that Elwing finally revealed his true abilities. In front of our very eyes, he leaped at one of the ghasts and turned into a bear before he'd even landed. It was amazing; I'd heard of shapeshifters, of course, and I knew that druids often could do this. But I'd never seen it done in real-life and I'd always thought it was only the most powerful of druids that could do this. Fortunately, he's not much smarter than the average bear, so Emmeral and I should have no problems taking him off-guard if need be.

Where did we go after this room? I remember we were set upon by ghouls along the hallways of the dungeon at some point, but nothing interesting was involved in that fight. I'm pretty sure that was when I needed to intimidate Emmeral into fighting alongside us, his possession having prevented him from being his usual self.

I'm incredibly glad I did this, since it saved my own life later. In attempting to intimidate him, I actually managed to terrify the possessing spirit out of him; with that, he was no longer possessed. He didn't explain to anyone about his possession, though, so I believe he didn't realize why he had been so afraid. Either way, he was ``cured'' now, and could continue his role of slaughtering our enemies and walking in front of us to search for traps.

At some point, we also came across a set of small bedrooms all nestled together. Each of these bedrooms contained something worthwhile; most of them contained gold or items worth some small amounts of gold, but we also came across a silvered shortsword, which Emmeral took since he was really the only one of us that could properly use it. He left his previous shortsword behind, which, at the time, seemed completely normal, but in retrospect was utterly idiotic. Why leave behind a perfectly good sword? I blame my possession for the fact that I didn't comment on this.

It was around this time that we started to make out the direction the chanting was coming from. Following our ears, we made our way deeper into the crypt.

At this point, I should mention that Kinnaeia got attacked by a door which she stuck her ear against to listen past; the door ended up being a mimic which held fast to her ear and attempted to hit her. Seeing a door attached to her ear attempting to hurt her\dots\ I must admit, I was laughing to hard to help her defeat it.

This mimic guarded the entrance to the final areas of this dungeon. This final level contained only two areas of note: a set of jail cells and a ritual room, a heavy portcullis locking it away.

Fortunately, we had with us all the finesse that is a bear.

Elwing, the bear, simply lifted the portcullis, throwing it upwards with such force that it actually broke. That was certainly convenient, as I'm not sure there was any other way into the room.

This room was the source of the chanting. We were confident of this, but somehow it fell completely silent as we entered. This scared us all enough that we explored the room incredibly cautiously, not even stepping on the central dias at first.

The room was swamped, with deep water throughout the room. A central dias sat in the centre, and a ledge around the outer wall allowed us to walk through it. At the back of the room, a heap of (what we thought was\dots) dirt and garbage sat in an alcove. We explored this cautiously, but eventually Scathac jumped onto the central dias. Immediately, cloaked figures appeared all around the ledge and begain their chanting again, this time proclaiming ``One must die''.

Kinnaeia shot an arrow at one of the figures, but it simply passed through, as if the figure wasn't even there.

This scared the ever-loving Hel out of us and we tore out of the room. All of us save Scathac, that is; as soon as she attempted to leave the dias, the refuse pile at the back of the room woke up and attacked her.

She fought it valiantly, but it ate her before any of us could return to fight against it. Fight against it we did, though, and we eventually killed it, though not before Elwing had come a hair's-breadth from death. With the creature's death, the chanting stopped and the cloaked figures vanished. The house had been cleansed of its evil.

We took some time to mourn our lost party member, though I admit I believe her death was for the best in retrospect. Between her actively working against me and her propensity to lead us blindly into danger, our lives our safer without her. Her death also gave Kinnaeia a chance to loot her body for the adventuring supplies she'd been missing. This was\dots\ surprisingly cold for a ``good'' cleric, but then, I've said before that something about her is not what it seems. I took Scathac's rope; you can never have enough rope.

Battered and bruised, we made our way back through the dungeon. And this is where I realized the other aspect of my possession: my ghost did not want me to leave the house.

I convinced the party to stay in the beds downstairs a night, to rest up before escaping the house. They listened to me, content to obey my wishes, while I silently tried to figure out how to escape my possession. I did not sleep well that night.

\sleep
