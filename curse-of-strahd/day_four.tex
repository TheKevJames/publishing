% The Ninth Day of the Ninth Moon, Year 735

Upon waking the next morning, we set out towards Vallaki, Irena in tow.

\section*{Towards Vallaki}
The path to Vallaki was surprisingly active. Or, maybe surprising is the wrong word. I'm not sure I'm surprised by much in Barovia anymore.

We'd barely walked an hour when we came across a fork in the road; an empty gallows ahead of us, between the forks, and a silent graveyard to the north. Any sane person would simply ignore these and walk past, but our cleric, of course, stopped to peer at it intently, searching for who-knows-what historical or religious tidbit. We paused, waiting for her, just long enough to wake the place.

As we walked away, a sound behind us alerted us to a dead body, swinging in the gallows, where it had not been only seconds before. It turned to look at us, its eyes dull and dead.

I said this place didn't surprise me much anymore? Yeah, I take that back, I was \emph{not} expecting the body.

Emmeral and Kinnaeia went to investigate the body, against our better judgement, but it disappeared in a puff of smoke as she approached. I have no idea what it was doing there. I'm not inclined to return to find out, either.

We hurried along, following the southern fork (remember: Vistani camp is probably down the northern road). Another hour and we came across the latest addition to our merry band of adventurers: Morianna.

It was as we walked through a dark forest, along a winding path, that we came across her. She looked terrible. Her back was to us, but even then we could see the weariness in her body, everything about her appeared bone-tired. The only thing keeping her awake, it seemed, was the group of wolves following her, barely being held back by her long and pointed stick.

The wolves didn't even notice us, so intent were they on following her, until Kinnaeia let fly an arrow at them. Somehow, despite her general inability to aim the thing, she managed to hit the wolf dead-on. Suddenly, then, all the wolves we on us, leaving Morianna alone.

We fought the wolves off, coming out of it not much worse for the wear, but the danger was there. I'm not sure how good of an idea it is to give a cleric a crossbow she can barely fire at the best of times and fires stupidly at the worst of times.

The woman we'd rescued introduced herself as Morianna. She's local to Barovia, she said, but she described her hometown as ``some place in the country, near Baratok'', though she wouldn't give the name of the town, at first. She refused to answer any questions about her past, even now as we set up camp and asked her again, but she asked us many questions about ourselves, what we were up to, where we were going, and why. She insisted she's never heard of Dwarves or Elves as anything other than mythical creatures.

She asked to accompany us, and Kinnaeia quickly jumped in to agree and fill her in on what we were up to. As we continued ahead, I fell back to walk with Emmeral.

I shared my suspicion of Morianna with him, and he seemed to agree. We briefly considered ditching them all, but I argued we were at least somewhat safer in a group -- something that is certainly true, despite Emmeral being our strongest warrior by far. We could likely set off on our own, but I'm not yet sure it would be worth the effort, nor would it be worth making an enemy of the others.

Perhaps in the future. I will be keeping my eye on them.

We continued ahead, eventually reaching another fork in the road. The fork to the east appeared ornate, with stone steps leading off toward what I strongly suspect is the castle above Barovia (town) -- Strahd's castle. I suspect we'll (or only some of us?) be returning back that way in the future.

Instead, we headed to the west toward Vallaki, through a set of huge gates and across a stone bridge. Across this bridge, two stone statues watched us from either side; the others appeared unconcerned, but I made sure to watch them carefully as we passed -- I wouldn'y put it past stone statues in this country to turn and follow us.

Surprisingly, nothing happened as we walked past the statues and through the gate. We simply continued ahead, along the road.

We passed a windmill to the west, its location matching the one I'd ``inherited'' from the Durst family. I said nothing to the others, and none of them considered it note-worthy. Given the magic present within the Durst house, I'd be unsurprised to find more secrets there; I'll need to return there, this time without a cleric or druid.

Speaking of druids, the road continued into a forest. A creepy forest, dark, and with trees that reached toward us to completely blocked out the sky. As we travelled, the forest murmured at us, growing more and more discontent. Eventually, I called out in Sylvanic that we meant no harm and simply wished to pass through. It was more out of annoyance than out of any real assumption that the trees around here spoke the same dialect as the trees in the Forgotten Realms.

Somehow, though, the trees understood me and responded without trying to kill us. I know, I was surprised too.

The trees calmed down and we soon ran across an old man sitting in the centre of the path. It turned out he was a druid, since Elwing immediately spoke to him in the flowing language of the Druids and he responded in kind. The man eventually left, and Elwing was left to translate.

Apparently, the old man is the keeper of that forest and was willing to let us pass only if we found ``his friend'' and convinced that friend to give us ``a token'' proving that the friend trusted us. Very vague, and nearly confusing enough to set us off track.

We turned back towards Barovia and grudgingly traced our path back out of the forest. Eventually, the sound of a horse racing around us alerted us to danger; we bunched up, watching the dark forest around us, only to find a skeletal man riding a horse, growing closer and closer to us with each pass.

We couldn't decide what to do about it; run, attack, or call out to him. Turns out we waited too long to decide; he same to a stop behind us, his weapon drawn.

I tried to talk to him, asked him who he was, what he wanted, why he was there. I couldn't get a response, he just stood there, not saying anything. Eventually, I gave up and started to continue walking away from him, convinced he was not the friend the druid had sent us for.

He seemed to be letting us go without any issue, when Kinnaeia turned back and asked him if he was killed by a vampire. He nodded, the first response we'd gotten so far. She asked if he was looking for a vampire. Another nod. She informed him that we were taking Irena to Vallaki to protect her from vampires, that we were gihting against Strahd, and that seemed to be enough for him. A raven flew to sit on her shoulder, and the death-like man rode away.

I wonder how Kinnaeia knew to ask him about vampires? It's a surprising leap of logic, in my mind; why would there be any connection bewtween some undead horseman and vampires, and why would that be the one thing he would respond to? Something's not right here\dots\ not to mention her declaring our allegiance against Strahd. Without any of us even discussing whether we were actually aligned against him, too.

Anyway, we continued on to Vallaki after that. The druid was no longer watching the path, and the trees were silent as we passed, so I guess the raven counted as the token we were meant to search for.

It turns out we weren't too far away from Vallaki; another fifteen minutes past the place we'd met the druid and we could see the town.

It looked grim. Stakes everywhere, with decaptiated wolves mounted on their tips. The walls themselves were made of fifteen-foot tall stakes, planted into the ground. Definitely the happiest of towns; when Emmeral suggested we camp outside of town for the night so we didn't have to approach this place in the dark, I couldn't have agreed more. Morianna was against that idea, but she didn't really have a good reason as to why. I'm not sure if our fears are unfounded, but I certainly don't want the city guards to mistake us for bandits and attack us.

I'm on first watch right now, along with Emmeral. The howl of wolves woke us earlier, but we couldn't find any of them near us, so we decided to ignore the sounds. The elves of our group won't have a problem with that, and Kinnaeia seems to have fallen asleep recently, but Morianna is staring into the sky unblinkingly. Clearly, she has a thing against wolves.

Anyway, with Emmeral keeping watch as well, I don't mind spending the time filling in this journal. I feel better already, like my mind is ordered better. Not to mention that writing all of this helped me set my thoughts straight. Let's see, I have my todo list:
\begin{itemize}
\item study spellbook
\item think about how my magic is developing
\item chat with Strahd
\item watch for suspicious behaviour from the others
\end{itemize}

Then there's the things we're doing as a group
\begin{itemize}
\item bring Irena to Vallaki, consider bringing her to Kresk
\item figure out how to escape Barovia?
\end{itemize}

I guess that second one should be on my list as well, but it's not a huge priority for me yet. Maybe that's what I should tell the others my top priority is, though; it seems a bit less suspicious than letting them figure out the other ones.

Either way, my watch is just about ending. I'll spend a bit longer studying this book, then I'll spend my trance thinking about my magic.

\pause

Thirteen pages. Nine with headings, but can't decipher headings. Some sort of code? Need to consider this. If each section is a spell, five one-page spells, four two-page spells. Level of complication? Think difference between \textsc{Fire Bolt} and \textsc{Magic Missile}. Sections have subheadings, looks like three is most? Subheadings in common! Three \emph{specific} subheadings, each spell has no more than those three. Section ``auyvkae'' has only text. Section ``lkuhkres kyugk'' has drawings, maybe maps? Section ``mvktu'' has diagrams; swirls, lines, arrows. What do the sections mean?

\sleep
