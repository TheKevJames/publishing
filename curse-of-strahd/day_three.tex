% The Eighth Day of the Ninth Moon, Year 735

In the morning, we continued out of the basement. The house had turned against us; poison spread through every room and the doors had been replaced by swinging blades. We hardly made it out alive, but a combination of quick reflexes and determining the pattern of the swinging blades helped us through.

Just as we neared the final door, though, my possession tugged harshly on me and I couldn't leave the house. I managed to make it obvious enough as to my predicament though, telling the others that I felt like this house was safe and worth staying in. It was just obvious enough, I think, that Emmeral finally realized what was going on. He yelled at me for a while, trying to scare out the ghost.

I've seen him kill with very little effort, slicing multiple monsters apart with a single swing of his sword. By rights I should be terrified of him, but somehow\dots\ he's just not a very intimidating person. It took him far too long to finally intimidate the ghost out of me. Though I'm thankful he did so, since it let me finally (\emph{finally!}) escape the house, I need to remember to never let him try to intimidate someone we need terrified. Useless.

\section*{Barovia}
With that, we stumbled into the light of Barovia -- the town, not the country.

Well, I say the light, but -- though the Fog was gone -- the gloom covering the sky prevented any real sunlight from reaching through to us. It was\dots\ brighter than nighttime, but not by much. This seems to be a trend in Barovia -- the country, not the town -- days later, and we've still yet to see the sun. A massive castle peered over a hill to the north, only barely illuminated by the lack of sun. We would discover lately, that the castle was Strahd's home.

We wandered around until we found a merchant near what we assumed was city centre. Emmeral bought some new armor, and didn't seem overly perturbed that the price was ten times as much as it would be anywhere else. He also kept asking about any expensive items the merchant may have had behind the counter. My assumption is Emmeral is richer than he's been letting on.

He did try to haggle a bit, though, but the merchant just called out his son (nephew?), a giant of a man that he used to strong-arm us into paying the normal fee. Emmeral quickly folded here, which makes sense; if someone is willing to threaten you that blatantly, they obviously won't appreciate any more haggling.

We asked the merchant if he had any maps, but he said he didn't. He gave us a vague description of the area, though, that I can basically summarize as: Barovia (town) is in the east part of Barovia (country). So incredibly useful, I know. I'm hopeful we'll be able to find a map tomorrow in Vallaki (I'll get to that later).

The merchant also directed us to the town bar. It was\dots\ not much to look at. Dirty, with the most boring barkeep I've ever come across: when we entered, he listed his wines and their prices in a monotone while wiping a glass with an even dirtier rag. Asking him a question just made him repeat the prices. I still have no idea what was going on with that.

Three colourfully-dressed women in the corner attracted Emmeral's attention, which lifted my spirits a bit: I enjoyed watching him fail to chat with the three women, their cold shoulders were hilarious to see. If I'd have thought they may have had any useful information, I may have joined them, but they didn't seem the brightest sort.

And sitting in the corner was Ismark, the burgomaster's son. We didn't know that at first, but he quickly noticed our ears (looks like elves are a rarity around here, lucky us), surmised we were adventurers, and asked for our help. His sister was being stalked by Strahd (yes, the same Strahd von Zarovich from the murder house), and he needed someone to bring her to safety in Vallaki.

We agree to come with him to talk with Irena (his sister), though we insisted we hadn't yet agreed to take him up on the quest. Even though we ended up agreeing, I must admit I was proud of us for not rushing into agreeing at the first hint of a request. Clearly, Scathac's unfortunate death left us less likely to act like heroes. I'm very much in favour of this.

As he brought us to his house, Ismark told us a bit more about the town; that he was the son of the town's burgomaster who had passed away only a few days ago and that the town was\dots\ well, not worth spending time in, between Strahd summoning wolves and other monsters and the ``March of the Dead'', both of which happened every night -- we didn't ask about the March, since we arrived at Ismark's house before we thought to ask. Another thing he mentioned: there's a Vistani camp to the west of Barovia. We've passed it now, but I suspect I know whereabouts it is (north at the fork with the gallows?) in case we need to head back there.

Upon reaching their house (which was in shambles, by the way. The attacks that Ismark had mentioned were obviously true at first glance, given the entire front yard was torn up), we found out that Irena didn't even want to leave the town. Part of this was that their father had not even been buried yet, but was laying dead in their house; the other was that she somehow liked this town and didn't want to leave, even for her own safety. Something about having grown up here.

We offered to provide protection for them and bring her father to the church for a proper burial, since it's about the only way we can convince her to leave. Even then, Irena barely felt safe enough -- this should have been proof enough for me that she was a coward, but I guess I missed the obvious. We carried the father in his makeshift coffin to the church; once there, Emmeral took to digging the grave while I watched over him and the body. The other's went to find the priest in order to perform a ceremony; personally, I thought it would be enough that we had a cleric with us, by Kinnaeia felt someone of ``the appropriate local faith'' should run the ceremony.

They returned shortly enough with the priest, who ran a simple enough ceremony that I'm confident I could have faked my way through. Ah well, it didn't take them much time to fetch the priest, so it's not worth complaining about. Besides, they did learn something interesting: the priest is hiding his son in the church basement, ever since his son was turned into a vampire.

This was our first indication of what we've stumbled into, by the way. I have a bad feeling about vampires, between the priest's son and Strahd\dots maybe focusing on learning magic that can deal with vampires would make more sense than wolves? We can manage these wolves, but vampires\dots\ hmmm.

Anyway, Kinnaeia and Elwing apparently decided to ignore the vampire, and Emmeral and I felt no need to complain, so we left the church after the funeral to return to the burgomaster's house for the evening and rest before we set out in the morning. As we left, the priest did inform us of a few more things: that the Abbey of Saint Markovia in Kresk (just past Vallaki) would be safer for Irena and that the March of the Dead we'd heard about involves hundreds of souls rising from the nearby graveyard and marching north every night.

Personally, I'm against bringing Irena to Kresk; so long as Vallaki has somewhere decently safe for her, there's no reason for us to continue with her to Kresk. I'd rather we search for some way out of Barovia (country, again) or, well, I'll write about my talk with Strahd in a minute.

As for the March? Well, it certainly piqued my interest at the time. Clearly something interesting was animating the dead -- and investigating that has gone well for me before. Unfortunately, we agreed that bringing Irena to safe haven took precedence, and so we didn't stay the night in Barovia to find out what exactly was going on.

On our way back to the burgomaster's house, Elwing proved that we hadn't completely removed the idiotic heroes from our group. We heard an old woman selling pastries the street over and were inclined to investigate. Nothing seemed amiss at first, and I was inclined to ignore the woman, but I waited too long to speak; trouble found us, regardless. A couple answered one of the door's the witch knocked on and the witch stole their child. They were, obviously, distraught, but made no move to get the child back. As the witch carried on her way, Elwing made his move.

Elwing charged out to confront her, the rest of us reluctantly following. Irena, the coward she is, ran home. Then again, is it cowardice if she made the correct decision?

Continuing the trend of my being our spokesman, I convinced her to let the child go. She seemed completely willing to do so, and unwilling to go to blows, but that wasn't enough for us. Elwing and Kinnaeia asked me to press the issue; I did, but I was nowhere near able to intimidate her into promising to never return. She simply laughed at me with a fire blazing in her eyes -- I knew then that she was simply playing with us and could wipe the floor with any one of us.

I was stunned with fear, which Elwing of course took as a good reason to attack her. Immediately, she let out a flurry of magic -- of perfectly controlled magic -- to devastating effect. It took Emmeral diving in and noticeably hurting her (along with the rest of our group hitting her and even Ismark helping out) before I was confident enough to join in. Even then, she focused her attention on me: after I launched a \textsc{Fire Bolt} at her, she simply laughed and launched one right back. Hers, mind, was significantly larger and more on-target than mine. I still remember her taunt: ``let me show you what \emph{real} magic looks like''. Oh the things I could have learned from her if we'd have simply tried to be on her good side instead of attacking her blindly.

It was a boost to my self-confidence later, when she shot out three \textsc{Magic Missile}s at me and I responded with four right back at her: this time, my magic responded well and I knocked her back with the force of my willpower. My attack, in fact, was enough to terrify her into running away.

Mindlessly, our group chased after her; Elwing and Emmeral sprinting after her, while I followed only closely enough to not lose sight of the others. Bloodthirsty morons.

Finally, we cornered her in an alley. The others surrounded her, slicing at her with their sword, but even that was not quite enough to fell her. They finally knocked her down and Emmeral questioned her about the area; she said nothing of importance, other than mentioning again the Vistani camp. Perhaps it is worth returning east to visit that camp? I hadn't realized it at the time, but the fact that both the witch and Ismark had mentioned it\dots\ maybe there is something of value there?

Emmeral -- in a rare moment of absolute stupidity -- answer the witch's question of whether we would trade her life for information by explaining that no, we would kill her either way. The witch, in a fit of anger, threw a flurry of spells at him, enough to knock him unconscious and give me some interesting ideas for spells. We eventually downed the witch, cutting her head off such that she disappeared in a blaze of dust, but hopefully the others will have learned a lesson from this.

Not that I think there's a big chance of that.

Having barely killed the witch that had been completely neutral to us, we continued on to Ismark's house, only to find Strahd standing in the front yard, staring into the sky.

\st{Strahd was} I don't know what to think about Strahd. Everyone acts as if he's the terrifyingly evil villain, but I'm not so sure.

He seemed\dots\ pleasant. I approached him and commented on the weather, and he seemed genuinely pleased at how nice it was (it was cold, cloudy, and rainy -- a perfect day for a vampire). He asked as to what we were up to, but didn't seem overly concerned that we were, technically, acting against him. He even asked as to what I, personally, wanted out of life.

He has power. He has knowledge. He has perfect control overhimself. And he has immortality. He has\dots\ everything I want and he knows that -- and he seems happy to share it with me. We didn't hammer out details or anything near that, especially not with the cleric breathing down my neck and Ismark's hand suspiciously on his sword, but he offered everything I desire and didn't even seem like he'd ask anything in return.

What would an immortal vampire with a country under his control ask for, anyway? He has anything he desires, what need would he have of asking something from me?

Power like his, I can't even believe it. He can control the weather across all of Barovia -- \emph{the entire country!} -- and bend it to his exact desires. He has long since discovered far more about the nature of magic than I, maybe than I ever will, and he would share that with me. He offered immortality, so I could continue my search infinitely.

I am incredibly tempted to take up his offer to join him.

Maybe he's evil? Maybe I'm evil for even considering it? I don't know. But what I do know is I need to know more.

The group still thinks I am searching for a way out of this Labelas-forsaken country, but that's only a ruse. Why would I want to escape this place when the target of all of my dreams is here? This doesn't mean I've given up my own search for the secrets of magic, not at all, but\dots\ well, I hope next time I speak with Strahd, there won't be an annoyingly nosey cleric watching me.

Seeing Ismark get back-handed into the side of the house when he finally lost his temper and attacked Strahd was funny, mind. If only it had happened to Kinnaeia.

\sleep
