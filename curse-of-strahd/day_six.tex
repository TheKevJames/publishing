% The Eleventh Day of the Ninth Moon, Year 735

The burgomaster came for us in the morning, after the guards had brought the other prisoner to the stocks. He took his time, mind, it was certainly late in the morning before he designed to speak with us.

Once again acting as our spokesman, I told him that the vampires were already present and that we'd been sent on a holy mission by the priest, though I was really very sorry that I couldn't break the priest's confidence on what the mission was.

Though he insisted we had no proof that the vampires were already there (apparently the coffin maker was found hanging, probably was killed by the vampires), but took my bait on the holy mission. He demanded to know what the mission was, so I ``grudgingly'' told him that since he was a respectable man he could be trusted with the knowledge of the priest's bones.

The flattery apparently did wonders, as did my insistence that the priest would back my story, since he let us free with a warning at that.

If I ever need to convince him of anything else he'd be against, it may be possible by referring to it as a favour for the priest. For some reason, the burgomaster seems to hold him in high regards; I'm sure I can abuse this.

Now I remember what it was I forgot. The priest's name, for the life of me I can't remember it. I wonder if Kinnaeia remembers? I'll have to ask her, just in case. \margin{Lucian Petrovich}

Turns out there may have been an easier way to convince him; he recognized Irena immediately upon setting us free and apologized profusely for locking her up (might be able to use this for future favours, too). Immediately, he invited us to breakfast with him and invited Irena to stay in his house, safe with him.

Now free, and with our equipment back (I need some way to hide my focus, at least, so it can't be taken away again), we joined him for breakfast.

\subsection*{The Burgomaster's Family Trouble}
Over breakfast, we were introduced to his wife; a demented woman, her only contribution to the conversation was to giggle mindlessly every time the burgomaster (\emph{Vargas Valakovich}) spoke and insist that ``all is well'' and her servant had ``no reason to leave the house -- all is well!''.

Irena formally accepted his offer for her to stay there, finally freeing us of our quest. The burgomaster, I notice, waited until after she'd agreed to inform us about his missing servants; apparently two of them had gone missing in the past while (week or two, I think), which explained the empty servant's beds Elwing had reported the night before.

Trying to show we were now on his good side, we agreed to help him find his missing servants; he was helpful here, at first, allowing us have free reign of his house to investigate, question any of his remaining servants, and answering all our questions truthfull (probably).

Following breakfast, we set off to explore; here, Elwing's knowledge of the maid's depressed look the night before was helpful -- upon bringing it up, she confirmed that ``all was \emph{not} well'', that the burgomaster's son \emph{Viktor} was a recluse that did not get along with the burgomaster, and that he may no more about the missing servants.

She directed us to his room, which we immediately headed toward.

The room was unlocked, but empty; nothing was out of place there. The room held a bookshelf, but none of the books appeared to be anything other than normal. Emmeral briefly examined a book of fairy tales, but it appeared to be normal. He certainly managed to hide his procilivities completely.

Continuing our exploration, we found several locked rooms; the only that we could enter without breaking in were the burgomaster's study and the master bedroom, where we found the entrance to the attic.

The study contained nothing useful, but the bedroom held some interest.

I need to work on my identification of magical objects. I'm sure there was something interesting about the mirror, if only I could have figured it out. A mirror in the burgomaster's bedroom caught our eye since it appeared to be a normal mirror for the others, but Emmeral and I saw ourselves aged immensely in the reflection.

Kinnaeia cast \textsc{Detect Magic} and informed us that the mirror was a source of conjuration magic. She also mentioned abjuration and evocation artefacts upstairs. Why she chose to remark on those, but not the necromantic magic, I have no idea. Maybe clerics aren't very good at identifying necromancy? I would assume it would be an important skill, but maybe I misunderstand; it could be that recognizing the opposite of one's own magic is impossible.

I'll think about that more, but more likely she just decided not to mention it for her own reasons. I don't trust her, obviously.

Either way, I \emph{really} need to learn how to \textsc{Detect Magic} like she can. Several times this has come up, my inability is inexcusable. Also some way to identify the properties of a magical object more in depth than simply its class of magic\dots\ but I'm getting ahead of myself.

The reason I need to learn how to better identify objects is that I accidentally broke the mirror; in some ways, I'm happy, since the method of breaking it proved that it may have been dangerous, but on the other hand\dots\ well, maybe we could have used the mirror for our own good.

I don't know if there's anything else I could have done without magic. I inspected the mirror and its frame, touched it, even, stared into my reflection. Nothing. Eventually, considering the conjuration magic, I wondered if I could try to interact with my reflection as if it were a real creature.

Visualizing a hand that could grasp at its shoulder, I cast \textsc{Chill Touch}. The mirror shattered, but in the moment before this, I'm sure I saw the hand grasp its shoulder but not mine -- I'm confident that is proof there was some being within the mirror. Whether it was an enemy or not, I'm unsure.

The \textsc{Chill Touch} is cast seems extraordinarily useful. The skeletal hand feels like necromancy, which I have never been able to cast before; this is probably incredibly useful against vampires, not to mention I feel like this has opened the entire field of necromantic magic up to me. I can't wait to learn more.

Elwing was less enthused by the broken mirror; he seemed genuinely angry that I'd broken it. Kinnaeia insisted I'd have to pay the burgomaster to repair the mirror, but having the burgomaster throw us out of his house fixed that problem.

The attic was crowded. A single foot path through the dirty main room forced us to walk in line; this turned out lucky for me, actually, since I was fourth in line. At the end of the path was a door, with the mark of abjuration Kinnaeia had mentioned earlier. I used \textsc{Mage Hand} to open the door, setting off a trap: lightning was shot at us, but fortunately Morianna raised her shield in time to prevent it from hitting most of us. Kinnaeia and Emmeral were less fortunate, being in front.

Immediately after opening the door, Emmeral charged in with his swords drawn, ready for a fight. Inside, though, we saw nothing; a study with books (that I would love to get my hands on), necromantically animated cat skeletons (that Kinnaeia still hadn't mentioned wreaked with necromancy), a rug over the evocation magic she'd identified earlier, and a chest in the corner.

Also, three creepy children in the corner of the room, staring at the wall and not moving.

We didn't have much time to look around, though; Kinnaeia reported some illusion magic in the corner, called out ``hello'', and then shrugged as if nothing was there.

Again, I really need to learn how to \textsc{Detect Magic}. If only I could have been watching the magic of the room myself, I'm sure I would have unerstood what had happened.

At this point, the cats attacked us; Kinnaeia helpfullly mentioning now that they were necromantic.

Morianna and Emmeral made short work of the cats, chasing them down even once they'd fled for us, though I have no idea why they were so insistent. The fight over (so we thought), I went to examine the chest in the corner.

Clothes. I have no idea why Viktor cared enough to put clothes in what otherwise appeared to be a treasure chest.

Suddenly, a \textsc{Fireball} hit us. I was completely knocked out by this; when the others lifted me up, I found that Kinnaeia had also been knocked out. Clearly this was a powerful \textsc{Fireball} cast by a powerful mage -- I wanted nothing to do with that.

As I began to flee, a bolt of ice flew into the room and demolished us again. Morianna seemed mostly unaffected by this, but apparently we were all knocked out. As she tells it, Viktor yelled at us to leave after casting the bolt and she snapped back that she was trying to heal us so we could leave.

Once we were all back on our feet, we sprinted away from Viktor's room. We seem to be sprinting away from fights a lot lately, clearly we need to fix something. I think in this case, Emmeral bursting into the room with his swords drawn couldn't have helped. We need to be less aggressive and avoid making people hate us.

As we fled the burgomaster's house, he (the burgomaster) caught up with us. He yelled at us for causing a disturbance, so I fired right back that Viktor was the one to blame. He got angry that we'd accused his ``perfect son'' and told us to leave his house. Since that was exactly what I was after, I consider that to be a good thing.

From there, we intended to head toward the inn to regroup.

On our way, though, we noticed we were being followed by an old man in a dark cloak. It's always one thing after another around here, I'm starting to get sick of it.

Still terrified because of Viktor, we didn't want to take any chances. We turned into an alley and Emmeral readied himself to grab the man. The rest of us had weapons drawn, waiting for the worst case scenario, whatever Labelas decided that would be.

Turned out we needn't have worried. The man claimed to be a servant of Lady \emph{Fiona Watcher}, who would ``rather serve the Devil than the burgomaster''. That's clearly an exaggeration, but either way: someone who hates the burgomaster than much may be a good ally to have. He told us the Lady wished to speak with us and that   we should go speak to her at our earliest convenience.

We told him we would consider it and let him go.

Back at the inn, we took some time to relax and re-gain our peace of mind. Rictavio was sitting in the same place he'd been before. He caught my eye as we entered, so I made sure to speak with him before we left the inn.

Morianna went to speak with some hunters in the corner, but apparently her talk provided no useful information. While she was speaking with them, though, we decided to leave this town and head for the Vistani camp nearby: hopefully they would be less likely to try to kill us, we thought. Turns out we were right, this time, which was nice. In fact, since leaving Vallaki today, no one has tried to kill us. It feels weird.

On our way out, I said my goodbyes to Rictavio. I made sure to imply we might come back for the festival to see his ``wonderful performance'', but that our ``wandering souls'' would likely lead us away.

Ugh, there's no way we're returning to Vallaki without the best of reasons.

\section*{The Vistani}
We left the city through a side gate nearest the Vistani camp and headed through the forest paths to their camp.

At first, I thought we'd found somewhere as bad as Vallaki, since the first two Vistani we saw were in the main tent, beating a young boy half to death. Turns out the kid had failed to watch the Vistani leader's daughter and was completely willing to be punished, which set my mind at ease.

It didn't make Kinnaeia feel better about the beating, though. She immediately stepped in and offered my assistance in interrogating the kid rather than beating him.

Yeah, she offered my assistance. Without asking me. Damned cleric.

With the two Vistani looking at me expectantly, I questioned the kid. It turned out he'd lost track of a girl named Arabella, the daughter of the man that'd been beating him (also the leader of the Vistani camp -- \emph{Luvash}). He wasn't watching her particularly closely, cause no one expected her to run off, so we didn't even have a starting point in looking for her.

Oh, did I not mention that? Yeah, Kinnaeia also offered our services in finding her. It turned out well for us what with the prophecy and all, but still\dots

Anyway, about this time Elwing switched back to being useful again; he told the men we had a dog that could track down the daughter by scent. Luvash got a hopeful look on his face at this and brought us to his wagon-house to give us something for ``our dog'' to smell.

Elwing and Morianna went in to the wagon while the rest of us waited outside. The golden hubcaps on the wagon look really expensive and Emmeral was certainly eyeing them speculatively\dots\ if he plans on taking them, I want in.

A few minutes later, they walked out of the wagon, waited a moment, Elwing transformed into a dog, and they re-entered the wagon. My best guess for their odd behaviour is that Elwing didn't want to reveal he could shape-shift, though that didn't last long. Uh, I'll explain soon.

Seconds later, Elwing (the dog) burst through the door of the wagon and started scampering north. Luvash yelled for his brother (the other man who'd been in the tent with us, named Arrigal) to join us and we set off.

We followed El-dog-wing north around Vallaki and toward Lake Zarovich. At first we were worried he was leading us back into Vallaki, but that didn't end up happening.

We found a few boats tied up on the shore and a lone fisherman sitting on a boat in the middle fo the like. As we stood, discussing our options (Dogwing had lost the scent at the shore), the fisherman stood at threw a burlap sack into the lake.

A burlap sack roughly the size and shape of a young girl.

Without pausing for thought, the six of us burst into action: Eldog leaped into the lake, transforming back into himself and then into a crocodile before hitting the water and beelining toward the sack; Emmeral and I jumped for the nearest boat and rowed with all our (mostly his) strength; and Kinnaeia, Morianna, and Arrigal did the same in a second boat.

El-croc lifted the sack out of the water, bringing it toward Arrigal's boat while we charged ahead to the fisherman. He made no move at any point, nor did he speak at all\dots\ it was eerie.

Behind us, Kinnaeia had pulled the little girl from the sack and was at work healing her. With a sputtering breath, she was eventually saved.

At this point, I realized we'd saved\dots\ maybe a vampire? I thought so at first, but I'm less sure now. Either way, something is weird about that girl. She was utterly calm at having just about died.

Having just woken from near death, Arabella explained her situation: she'd been taken by the man and had been in that bag ever since. She did this calmly, impassively\dots\ maybe she is a vampire?

Well, at the time, I thought she might have been a vampire and figured we should stay on her good side, just in case. I deferred to her on what she would like done with the fisherman.

She very calmly informed us she did not care about him and that she really only cared about returning to the Vistani.

With a nod, I ordered Kinnaeia and Morianna to take her to shore. Arrigal jumped over to our boat and we pulled alongside the fisherman.

No matter what we asked, the fisherman said nothing, made no move, and simply continued to hold his fishing rod in the lake. Eventually, with a shrug, Emmeral lended his sword to Arrigal. Arrigal stabbed the fisherman\dots\ a dozen times, maybe? He clearly has some anger issues, but then, the man had just stolen his neice.

The fisherman had no reaction, even to getting stabbed. Maybe he was in some sort of magical trance? Yet another reason I need to learn how to \textsc{Detect Magic}.

We brought his boat back to shore and brought the girl home, Morianna cradling her in her arms the entire way. Along the way, Crocwing reverted himself to his Elven form and explained his shapeshifting to Arrigal.

``That sounds useful,'' Arrigal said, clearly overwhelmed and a bit confused.

As we returned to the camp, we found a large crowd of Vistani waiting for us, along with what we later found out was a Dusk Elf.

An \emph{elf}! Here, in Barovia! I'm starting to doubt Morianna's claim that she'd thought we were mythical creatures; Dusk Elves live here in the camp, Rictavio himself is an elf, \dots\ we've been here only half a week and we've already come across several Elves other than ourselves.

Not that the elf was much interested in conversing with us.

Luvash, of course, was overjoyed that we'd recovered his daughter and threw a feast in our honour. He had his man bring out dozens of casks of wine -- all labelled with a wizard hat, like the prophecy described -- and everyone went wild.

Kinnaeia challenged Arrigal to a drinking competition. Utterly foolish, since she passed out after two flagons, but pretty funny to watch. For us, at least, I'm sure she'll have a wicked headache tomorrow.

The rest of us were approached by Luvash shortly after that; he brought us to a locked weapon containing all the Vistani's wealth and told us to ``take anything''.

Well, we're not going to say no to an offer like that.

We all headed for the jewelery box first; everything looked gaudy to me, but Morianna took a necklace to give to Kinnaeia. Elwing grabbed a flask which Luvash described as ``some sort of illusion magic''. He later gave this to me because ``I like magic''\dots\ I guess he's trying to get on my good side, but I have no idea why. It seemed so out of the blue\dots

Emmeral and I went for more traditional wealth, pulling coins from a wooden chest and an iron chest with gold and electrum filling them.

Thanking Luvash for his generosity, we left to re-join the party. After a while, we decided to see what we could learn from the elf.

As we approached his house, we noticed he'd been watching us through the window. He exited the house as we arrived, speaking with us outside.

I greeted him as a friend, showing my Elvish ear with the same motion I'd learned from Rictavio, but he didn't seem to care -- he was rude, abrupt, and wanted nothing to do with us.

Morianna pried a bit, after he'd told us to leave, but all that accomplished was making him angry. We left after that; after I'd apologized for our rudeness. Elwing spoke at the elf in Druidic, apparently looking for a response, but the elf only gave him a blank look.

We headed back to the party. I noticed Emmeral peeled away from us to head back to the elf's house, but he later reported he'd learned nothing interesting. The night passed without much of interest after that -- we learned that Elwing can play the flute he'd purchased the day before as he joined in with the band.

It was interesting to watch for a while, but eventually I left for the trailer Luvash had offered us for the night to study the spellbook for a while.

Eventually the others joined me and we went to sleep for the night, only to be woken what felt like minutes later by the prophetic Arabella.

Alright, now I'm caught up. It's been an hour or so since she delivered the prophecy and it's starting to make sense to me.

Here's what I think the prophecy means:
\begin{itemize}
\item Strahd is either working against us, or will be, or is or will work against at least some of us
\item somewhere in the Wizards of Wine winery we'll find information on Strahd's past
\item in a marsh (near the town to the south, probably), we'll find someone important that can direct us to a young woman, who either is or has some holy protection for us
\item a sword made out of sunlight can be found in the haunted house where the dragon used to live
\item a female werewolf that hates Strahd can be convinced to help us
\item there is an immortal dark creature who lives near the tomb of a man he envied
\end{itemize}

These seem to be scattered all across Barovia. Once the others awake, we'll definitely be discussing which of these we should head to first. If any, that is, there's also the option of returning to Vallaki\dots\ I don't really want to return to the city, but then, the Madam Watcher character sounds like the sort of person we don't want to anger by refusing her request to meet\dots\ I don't know. Hopefully, the others will have some insight.

\pause

``mvktu'' section swirls in common between pages? Bottom of page four, top of page nine. Something about swirl pattern on five looks familiar\dots\ why? Try notations as hand movements? Not both hands. Not left, right still. Not right, left still. Visualize? No. Need focus? Focus in left, move right hand, feel something. Magic is responding?

Movements on page five pulls my magic up, feels like the moment before a normal spell before fading. Need some release on this spell. Spell has ``auyvkae'' section, required? Maybe is verbal component? Why do other section diagrams have no effect when doing same? Spells with ``lkuhkres kyugk'' make sense, maybe need something else, but with only ``auyvkae'' as well should have same effect\dots

\textsc{Magic Missile}. I knew it, I knew I recognized it! As soon as I let the movements become more relaxed, it just came to mind: I was making the same movements with my hands that my magic makes inside me when letting out \textsc{Magic Missile}s. As soon as I realized this, my magic responded.

If this is how this spellbook records magic, I can figure it out.

\textsc{Hold Person} and \textsc{Invisibilty} are in this book. I knew how to case them already, but this is giving me more to work with. Both had ``lkuhkres kyugk'', I wonder if this is related to them only working while I'm holding my focus? Maybe they describe physical spell components\dots

I've done it! This entire book, I understand it all! I can cast any of them, I understand! \textsc{Disguise Self}, \textsc{Identify}, \textsc{Mage Armor}, \textsc{Magic Weapon}, \textsc{Protection from Good and Evil}\dots\ everything in this book will be useful. And now that I've deciphered one spellbook, I'm sure I could do it again.

I need to find another spellbook and find out what is the same between wizards. Already I've learned so much: that magic has forms, each spell representing itself differently in shape, in feel, in how it moves. My magic can shape itself into the right form to suit my desires, sometimes, but now\dots\ now I can help it along, bound it to the shape of the spell I want rather than relying so much on luck.

I feel like I have so much more control over it, that I can learn to bend it exactly to my will. I'm exhausted now, but tomorrow\dots\ I hope I get a good excuse to exercise my knew powers, to figure out exactly what my new limits are.

\sleep
